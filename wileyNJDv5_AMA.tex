% \documentclass[AMA,Times1COL]{WileyNJDv5}
\documentclass[LATO2COL]{WileyNJDv5}

\articletype{Original Research}%

\received{Date Month Year}
\revised{Date Month Year}
\accepted{Date Month Year}
\journal{Journal}
\volume{00}
\copyyear{2025}
\startpage{1}

\raggedbottom
\usepackage{hyperref}
\usepackage{unicode-math}
\usepackage{fontspec}
\usepackage{graphicx}
\usepackage{enumitem}
\usepackage{xcolor}
\usepackage{booktabs}
\usepackage{float}
\setmainfont[
  BoldFont={Arial-BoldMT.ttf},
  ItalicFont={Arial-ItalicMT.ttf},
  BoldItalicFont={Arial-BoldItalicMT.ttf},
  Path=./Fonts/Arial/
]{Arial.ttf}

\setmathfont{Latin Modern Math}
\begin{document}

\title{A decentralized application based on blockchain for insurance policy sales}

\author[1]{Alex Azevedo do Prado}
\author[1]{Eduardo Takeo Ueda}
\author[1]{Anderson Aparecido Alves da Silva}
\author[2]{Vilc Queupe Rufino}

\authormark{PRADO \textsc{et al.}}

\titlemark{A decentralized application based on blockchain for insurance policy sales}

\address[1]{\orgname{Instituto de Pesquisas Tecnológicas do Estado de São Paulo}, \orgaddress{\state{São Paulo}, \country{Brazil}}}

\address[2]{\orgname{Marinha do Brasil}, \orgaddress{\state{São Paulo}, \country{Brazil}}}

\corres{Alex Prado, Instituto de Pesquisas Tecnológicas do Estado de São Paulo \email{alex.prado@ensino.ipt.br}}


\abstract[Abstract]{The insurance industry continues to face persistent challenges related to fraud detection and prevention, as well as risks arising from opportunistic and malicious actors. Insurance policies function as financial risk-transfer instruments designed to compensate losses caused by adverse events, such as accidents, property damage, illness, or death.
To address these challenges, insurers are increasingly adopting emerging digital technologies, particularly blockchain, to enhance transparency, security, auditability, and operational efficiency across the policy lifecycle. Blockchain-based architectures, combined with smart contracts for the automated execution and enforcement of policy terms, are gaining momentum, with their commercial value projected to reach approximately USD~3.1~trillion by 2030. Smart contracts enable deterministic execution of contractual rules and have become increasingly relevant in digital economy and smart industry applications.
Despite their potential, large-scale adoption remains constrained by technical limitations, including low transaction throughput, block size restrictions, ledger growth, and cryptographic overhead. In this context, this work proposes a solution based on a private blockchain architecture employing smart contracts to support automated decision-making in the insurance policy sales process, considering contractual relationships between insurers and policyholders.
The research methodology comprises two stages: a systematic literature review to establish a comprehensive understanding of blockchain-based solutions in insurance, followed by an experimental study with a quantitative approach to implement and evaluate the proposed system. The expected contributions include advancing the literature on smart contract-driven decision-making in insurance and presenting a practical, secure, and auditable blockchain-based solution for automating insurance policy sales.}


\keywords{Blockchain, Smart contracts, Insurance Policy, Decision-making, Insurance Sales}

\jnlcitation{\cname{%
\author{Alex P.},
\author{Eduardo E},
\author{Anderson S}, and
\author{Vilc R}}.
\ctitle{On simplifying ‘incremental remap’-based transport schemes.} \cjournal{\it J Comput Phys.} \cvol{2021;00(00):1--18}.}

\maketitle

\renewcommand\thefootnote{}
\footnotemark\footnotetext{\textbf{Abbreviations:} TPS,  transactions per second; APC, antigen-presenting cells; IRF, interferon regulatory factor.}

\renewcommand\thefootnote{\fnsymbol{footnote}}
\setcounter{footnote}{1}

\section{Introdução}\label{sec1}

O método convencional de documentar informações do seguro contratado é frequentemente inseguro, ineficiente e suscetível de manipulação. No entanto, uma vez documentados, os dados devem permanecer inalteráveis e resistentes à manipulação. Esta característica é essencial no contexto do seguro, onde a exatidão e integridade dos registros desempenham um papel fundamental na resolução de litígios e na garantia da confiança entre as partes interessadas \cite{Sodhi2024}.

O sistema brasileiro de consulta a registros de seguros, destinado a usuários na qualidade de segurados, foi instituído com o objetivo de viabilizar a verificação da existência de certificado ou apólice de seguro emitidos por empresas autorizadas pela Superintendência de Seguros Privados (SUSEP), vinculados ao Cadastro de Pessoa Física (CPF) do consulente. Todavia, o referido sistema não contempla informações relativas a planos e seguros de saúde, à previdência complementar aberta, a títulos de capitalização, a seguros cujo período de vigência já tenha se encerrado, a contratos firmados há menos de quatro dias úteis, bem como a determinados seguros de pessoas, como, por exemplo, seguros de vida e de acidentes pessoais \cite{govbr2024}.

As empresas recebem uma crescente pressão para desenvolver modelos de negócio mais sustentáveis para um mundo em constante mudança, para criar valor, não apenas para os acionistas, mas também para consumidores e população afetada \cite{Baldassarre2017}.

Desde a primeira revolução industrial, muitos engenheiros tentam resolver problemas relacionados às operações e, ao fazê-lo, tentam melhorar a eficiência dos processos produtivos. Até 2050, a população do planeta está destinada a atingir 9,3 bilhões e diferentes desafios e questões estão aumentando a pressão sobre a indústria atual \cite{ElHamdi2019}.

A indústria depende de vários processos entre as partes para iniciar, manter e encerrar diferentes transações. Consequentemente, o tempo de processamento das transações, a liquidação, o tempo de pagamento e a segurança da execução do processo são as principais preocupações \cite{Amponsah2021}.

A construção de um aplicativo na blockchain, no qual cada \textit{node} é conectado a todos os outros \textit{nodes} pela rede, pode ajudar a eliminar os problemas de segurança, levando em conta as características inerentes da blockchain em conjunto com os mecanismos de consenso \cite{Roriz2019}. 

Ao fazer parte da rede blockchain, cada seguradora tem uma cópia dos dados e do código, eliminando um único ponto de falha que poderia afetar toda a rede, pois se tornariam como computadores interagindo entre si na mesma rede. \cite{Roriz2019}

A Figura \ref{fig:fig1}, a seguir, ilustra o sistema de uma seguradora é representado por um \textit{node} da rede blockchain. Quando é criada uma apólice de seguro, ela primeiro deve passar por todos os critérios presentes no contrato inteligente e só então a transação pode ser submetida ao blockchain. Na solução proposta, todas as seguradoras devem fazer parte da rede blockchain \cite{Roriz2019}.

\begin{figure}[h]
    \centerline{\includegraphics[width=0.85\linewidth]{images/Figura1-ArquiteturaBlockchain.jpg}}
    \caption{Arquitetura de solução blockchain para seguradoras}
    \label{fig:fig1}
\end{figure}

Em caso de emergência, o seguro é uma das assistências essenciais acessíveis às populações para neutralizar os seus custos e assisti-las. O maior desafio do setor é como detectar e proteger fraudes e impedir as más intenções dos falsos participantes. Desta forma, o impacto positivo da tecnologia blockchain foi observado pelas principais seguradoras e resseguradoras, resultando em investimento em sistemas experimentais \cite{Amponsah2021}.

As seguradoras que oferecem produtos que possuem coberturas para bens como o imóvel ou o automóvel, poderiam reunir os dados de dispositivos conectados e analisá-los para um gerenciamento preciso durante o processo de sinistros e desta forma impedir as más intenções de falsos participantes \cite{Xiao2020}.

O diagrama apresentado na Figura \ref{fig:fig2}, ilustra o caso de uso envolvendo três tipos de participantes, sendo os motoristas de veículos, os operadores de transporte, e as companhias de seguros. A interação e a relação entre estes participantes exigem a implantação de dispositivos IoT (Internet of Things) a partir dos quais é possível realizar a coleta dos dados onde são processados pelo Sistema de Informação Geográfica para que seja possível a análise e posteriormente a gestão do processo de seguros \cite{Xiao2020}.

\begin{figure}[h]
    \centering{\includegraphics[width=1\linewidth]{images/Figura2-TransportInsurance.png}}
    \caption{Caso de uso de aplicação de seguro de transporte}
    \label{fig:fig2}
\end{figure}

Os contratos inteligentes podem encontrar potenciais cenários de aplicação na economia digital e indústrias inteligentes, incluindo serviços financeiros, saúde e IoT, entre outros, e estão integrados às principais plataformas de desenvolvimento baseadas em blockchain, como Ethereum (ETH). O ETH é uma rede pública, um blockchain e um protocolo de código aberto, operado, governado, gerenciado e de propriedade de uma comunidade global de dezenas de milhares de desenvolvedores, operadores de nós, usuários e detentores de ETH \cite{Ethereum2024}. Já o Hyperledger Fabric que serve como base para o desenvolvimento de aplicações ou soluções com arquitetura modular permite componentes intercambiáveis, incluindo serviços de consenso e adesão, possibilitando um ambiente plug-and-play \cite{HyperledgerFoundation2024}.

O objetivo principal deste artigo é propor o uso do blockchain e contratos inteligentes na indústria de seguros na etapa de venda de uma apólice de seguros. Para isso, apresenta uma implementação de uma aplicação descentralizada baseado em blockchain aplicado em operações de vendas de apólices de seguros e contratos inteligentes para registro das transações mapeando as melhores práticas observadas e assim contribuindo para aplicação na indústria. 

\section{Trabalhos correlatos}\label{sec2}

A adoção de tecnologias baseadas em blockchain e contratos inteligentes é amplamente reconhecida como um fator de transformação no setor de seguros, com potencial para automação, aumento da transparência e redução de custos operacionais \cite{Shetty2022,Aleksieva2020,Xiao2020}. Essas tecnologias possibilitam o registro imutável de informações e a execução automática de regras contratuais, reduzindo a dependência de intermediários. Plataformas como Ethereum e Hyperledger Fabric têm sido empregadas em diferentes contextos, oferecendo flexibilidade para o desenvolvimento de aplicações descentralizadas no domínio de seguros \cite{Aleksieva2020c,LIU2019}.

Apesar do interesse crescente, a adoção dessas soluções enfrenta desafios técnicos e organizacionais, incluindo integração com sistemas legados, limitações de escalabilidade, conformidade regulatória e adaptação de seguradoras, corretores e clientes a ambientes digitais \cite{Aleksieva2020b,Bai2022}.

\subsection{Desafios e Oportunidades na Indústria de Seguros}\label{sec2.1}

O setor de seguros passa por uma transformação contínua impulsionada por inovações tecnológicas, exigindo das seguradoras soluções que garantam competitividade e eficiência operacional. Tecnologias como blockchain e inteligência artificial são apontadas como estratégicas nesse contexto. A blockchain, enquanto banco de dados distribuído, oferece transparência, rastreabilidade e imutabilidade, além de viabilizar a automação de processos por meio de contratos inteligentes \cite{Shetty2022}.

Estudos indicam que a blockchain pode favorecer o desenvolvimento de novos produtos, aprimorar a detecção de fraudes e aumentar a confiabilidade das informações registradas. Aleksieva \cite{Aleksieva2020b} destaca seu uso para tornar os processos de compra e venda de apólices mais eficientes, reduzindo custos e ampliando a transparência. Outros trabalhos discutem sua aplicação na automação de processos e na mitigação de problemas de confiança no setor \cite{Xiao2020}. No contexto de seguros agrícolas, a integração entre blockchain e IoT é explorada para reduzir erros em pagamentos de sinistros \cite{Bai2022}, enquanto Liu \cite{LIU2019} aborda sua aplicação em seguros de viagem. Estudos adicionais ressaltam o uso de blockchains permissionadas para atender a requisitos de privacidade e controle de acesso \cite{Aleksieva2020c,Klapkiv2022}.

\subsection{Aplicações de Blockchain e Contratos Inteligentes na Indústria de Seguros}\label{sec2.2}

A literatura apresenta diversos casos de uso da blockchain no setor de seguros, abrangendo desde seguros de viagem e vida com pagamentos automatizados até a automação de resseguros, gerenciamento de sinistros, digitalização de identidades e seguros peer-to-peer \cite{Shetty2022}. Os contratos inteligentes são apontados como elementos centrais para a automação de processos, permitindo a execução automática de cláusulas contratuais e a redução de disputas.

Aleksieva \cite{Aleksieva2020} descreve aplicações da blockchain no processamento de sinistros, citando plataformas como Corda e Etherisc, enquanto Aleksieva \cite{Aleksieva2020b} explora iniciativas como Nexus Mutual, Insureum e Savemyluggage. Outros estudos analisam a integração entre blockchain e IoT, destacando benefícios como automação de pagamentos e detecção de fraudes, especialmente em domínios específicos \cite{Bai2022}. Liu \cite{LIU2019} enfatiza o uso da blockchain em seguros de viagem com foco na integridade e proteção de dados transacionais.

\subsection{Implementação de Blockchain e Contratos Inteligentes}\label{sec2.3}

A implementação de soluções baseadas em blockchain e contratos inteligentes requer algoritmos criptográficos e processos digitais avançados para garantir segurança, autenticidade e integridade das transações. Os contratos inteligentes caracterizam-se por sua representação digital, incorporação de cláusulas contratuais em código e execução automatizada condicionada ao cumprimento das regras definidas \cite{Shetty2022}.

Plataformas permissionadas, como o Hyperledger Fabric, são apresentadas como adequadas para ambientes corporativos e consórcios empresariais, devido à sua flexibilidade e capacidade de adaptação \cite{Aleksieva2020}. Outros trabalhos discutem a integração entre blockchain e IoT para seguros de veículos, possibilitando modelos de precificação dinâmica \cite{LIU2019}. Estudos adicionais destacam os impactos positivos da blockchain na automação de processos internos, como pagamento de sinistros e prevenção de fraudes, bem como a influência do porte organizacional e do momento de adoção nos benefícios obtidos \cite{Bai2022,Klapkiv2022}.

\subsection{Escalabilidade no Processamento de Transações da Blockchain}\label{sec2.4}

A escalabilidade constitui um dos principais desafios para a adoção de blockchain no setor de seguros. Contratos inteligentes podem impulsionar a inovação, automatizar pagamentos e melhorar o relacionamento com clientes \cite{Shetty2022}, além de otimizar processos de resseguro e operações de corretoras \cite{Aleksieva2020}. No entanto, blockchains públicas, como Bitcoin e Ethereum, apresentam limitações de desempenho em termos de transações por segundo, o que pode comprometer aplicações com alto volume transacional.

Nesse contexto, blockchains privadas ou permissionadas, como o Hyperledger Fabric, são frequentemente apontadas como alternativas mais adequadas para cenários que exigem maior eficiência transacional \cite{Xiao2020}. A integração entre blockchain e IoT é considerada promissora para aprimorar escalabilidade e privacidade, sendo discutidos modelos híbridos que combinam diferentes plataformas \cite{Xiao2020,Aleksieva2020b}. Estudos também destacam o uso de IoT para coleta de dados em tempo real e a importância de mecanismos robustos de segurança e rastreabilidade para mitigação de fraudes \cite{Bai2022,LIU2019}. A evolução das tecnologias de blockchain e seu potencial para reduzir assimetrias de informação e custos de transação no setor de seguros são analisados em trabalhos recentes \cite{Aleksieva2020c,Klapkiv2022}.

\subsection{Análise Crítica dos Trabalhos Correlatos}

A literatura concentra-se predominantemente na automação de sinistros, na rastreabilidade de dados e na redução de custos operacionais, dedicando atenção limitada às fases iniciais do ciclo de vida do seguro, especialmente à comercialização e emissão de apólices. Embora contratos inteligentes sejam amplamente mencionados, poucos estudos detalham sua aplicação na automação integrada de propostas, negociações e formalização de contratos.

Observa-se também uma lacuna quanto à avaliação empírica da integração entre plataformas blockchain e sistemas legados de seguradoras e corretoras, bem como à definição de mecanismos de governança e gestão de identidades em ambientes multiatores. Além disso, aspectos relacionados ao controle de acesso e à segurança entre diferentes perfis de usuários permanecem pouco explorados.

As questões de escalabilidade são frequentemente tratadas de forma genérica, sem foco nos requisitos de desempenho associados à venda de apólices em tempo real. Tecnologias como redes híbridas e mecanismos avançados de escalabilidade são citadas, mas raramente avaliadas no contexto específico da emissão e venda de apólices. Diante dessas lacunas, este trabalho propõe uma aplicação descentralizada baseada em blockchain permissionada para a automação segura e auditável da venda de seguro viagem, integrando controle de acesso, verificação de integridade dos dados e soluções de escalabilidade compatíveis com as exigências operacionais do setor.


\section{Definição da Arquitetura Descentralizada}\label{sec3}

Esta seção descreve a aplicação da tecnologia blockchain no domínio de seguros, com foco específico no processamento da venda de apólices por meio de contratos inteligentes e no registro descentralizado das transações.

O diagrama de arquitetura apresentado na Figura~\ref{fig:fig3} adota uma abordagem descentralizada e foi projetado para suportar uma aplicação baseada na plataforma Hyperledger Fabric, voltada à comercialização de apólices de seguro. Essa perspectiva técnica permite identificar, de forma clara, os principais componentes, camadas e fluxos envolvidos na validação e no registro das transações, assegurando propriedades essenciais ao setor de seguros, como transparência, segurança e rastreabilidade.

A arquitetura especifica as interações entre a aplicação cliente, a infraestrutura Hyperledger Fabric, os contratos inteligentes e os mecanismos de validação distribuída, evidenciando como esses elementos se integram para garantir a confiabilidade e a integridade do sistema. Os principais componentes são descritos a seguir.

\begin{itemize}
    \item \textit{Insurance-Sales-API}: interface da aplicação responsável pelo gerenciamento das transações relacionadas à venda de apólices de seguro. Esse componente atua como ponto de entrada para usuários e sistemas externos, permitindo a comunicação com a rede blockchain.
    
    \item \textit{Gateway}: módulo intermediário entre a API de vendas e a infraestrutura Hyperledger Fabric, composto por contratos inteligentes associados aos peers \textit{peer0-org1} e \textit{peer0-org2}. Esses contratos encapsulam a lógica de negócio necessária para validação e registro das transações na blockchain.
    
    \item \textit{Organizações da Rede}: a rede Hyperledger Fabric é composta por duas organizações, cada uma com funções específicas:
    \begin{itemize}
        \item \textit{Organização 1 (Org1 -- Seguradora)}: representa a seguradora responsável por definir e gerenciar as regras de elegibilidade e validação das apólices. O peer \textit{peer0-org1} mantém o ledger da organização e executa os contratos inteligentes relacionados às políticas da seguradora.
        
        \item \textit{Organização 2 (Org2 -- Corretor de Seguros)}: representa o corretor de seguros, responsável pela intermediação das vendas. O peer \textit{peer0-org2} mantém uma cópia do ledger e executa contratos inteligentes voltados à validação e à intermediação das transações.
    \end{itemize}
    
    \item \textit{Infraestrutura Hyperledger Fabric}: compreende os elementos centrais da rede blockchain, incluindo:
    \begin{itemize}
        \item \textit{Peers} (\textit{peer0-org1} e \textit{peer0-org2}): responsáveis pela validação e armazenamento das transações no ledger distribuído, bem como pela execução dos contratos inteligentes;
        \item \textit{Orderer}: componente responsável pela ordenação e disseminação das transações na rede, garantindo consistência e confiabilidade;
        \item \textit{Certificate Authority (CA)}: autoridade certificadora encarregada da emissão e do gerenciamento das identidades digitais dos participantes da rede;
        \item \textit{CouchDB}: banco de dados utilizado como repositório de estados dos peers, possibilitando consultas avançadas aos dados armazenados no ledger.
    \end{itemize}
    
    \item \textit{Hyperledger Explorer}: ferramenta destinada ao monitoramento e à visualização das informações registradas na rede blockchain, composta por:
    \begin{itemize}
        \item \textit{Explorer Console}: interface gráfica para visualização de blocos, transações e topologia da rede;
        \item \textit{Explorer DB}: banco de dados utilizado pelo Hyperledger Explorer para armazenamento e consulta das informações coletadas.
    \end{itemize}
\end{itemize}

\begin{figure}[h]
    \centering
    \includegraphics[width=1.00\linewidth]{images/proposalarchitecture.png}
    \caption{Diagrama de arquitetura da aplicação proposta}
    \label{fig:fig3}
\end{figure}

A rede blockchain foi implantada utilizando contêineres, o que proporciona flexibilidade para a configuração e o gerenciamento da infraestrutura, permitindo ajustes como:
\begin{enumerate}[label=\alph*)]
    \item definição da localização lógica dos nós da rede;
    \item alocação de recursos computacionais, como CPU e memória;
    \item configuração dos peers endossadores necessários ao consenso;
    \item inclusão de novas organizações e participantes na rede.
\end{enumerate}


\subsection{Etapas do Fluxo de Dados}

O diagrama de fluxo de dados apresentado na Figura~\ref{fig:fig4} ilustra o processo de aquisição de uma apólice de seguro, desde a solicitação inicial do usuário até a confirmação da transação. O objetivo é fornecer uma visão clara e estruturada de como a compra da apólice é processada, validada e registrada no sistema conforme as etapas do fluxo são descritas a seguir:

\begin{enumerate}[label=\alph*)]
    \item \textit{Solicitação da apólice}: o usuário inicia o processo de compra por meio da aplicação cliente, que envia a solicitação à \textit{insurance-sales-api};
    
    \item \textit{Encaminhamento ao Gateway}: a API de vendas encaminha os dados da apólice ao Gateway, responsável por direcionar a transação à organização apropriada;
    
    \item \textit{Execução dos contratos inteligentes}: o Gateway aciona os contratos inteligentes executados nos peers \textit{peer0-org1} e \textit{peer0-org2} para validação das regras de negócio;
    
    \item \textit{Processamento da transação}: caso os critérios definidos nos contratos inteligentes sejam atendidos, a transação é encaminhada ao \textit{Orderer}, que realiza sua ordenação e inclusão no ledger. Caso contrário, a solicitação é rejeitada;
    
    \item \textit{Armazenamento dos dados}: os peers registram as informações da apólice no ledger distribuído e no banco de dados \textit{CouchDB}, viabilizando consultas posteriores;
    
    \item \textit{Monitoramento da transação}: seguradora e administradores podem acompanhar o status da transação por meio do \textit{Hyperledger Explorer};
    
    \item \textit{Confirmação ao usuário}: após o registro da transação na blockchain, a API notifica o usuário quanto à conclusão do processo de compra.
\end{enumerate}

\begin{figure}[h]
    \centering
    \includegraphics[width=1\linewidth]{images/dataflow.jpg}
    \caption{Diagrama de fluxo de dados}
    \label{fig:fig4}
\end{figure}

O fluxo apresentado na Figura~\ref{fig:fig4} evidencia as interações entre os principais componentes do sistema, incluindo a API de vendas, o Gateway, os peers da rede blockchain, o Orderer, o \textit{CouchDB} e o \textit{Hyperledger Explorer}. Cada etapa demonstra como os dados são transmitidos, validados e registrados de forma distribuída, assegurando a confiabilidade, a integridade e a rastreabilidade das informações relacionadas à apólice de seguro.


\subsection{Adoção do Hyperledger Fabric}

A adoção do Hyperledger Fabric neste trabalho baseia-se no estudo de Androulaki \cite{Androulaki2018}, que define o Hyperledger Fabric como uma plataforma blockchain permissionada, projetada para a execução de aplicações distribuídas escritas em linguagens de programação de uso geral, como Go, Java e Node.js. A plataforma permite o rastreamento seguro do histórico de execução das transações por meio de estruturas de dados distribuídas e não possui criptomoeda nativa, característica que a torna adequada para ambientes corporativos.

O Hyperledger Fabric possibilita a integração de diferentes domínios de aplicação, incluindo Internet das Coisas, saúde e seguros, por meio do desenvolvimento de contratos inteligentes, denominados \textit{chaincode}. Conforme ilustrado na Figura~\ref{fig:fig5}, o \textit{chaincode} atua como a camada de lógica de negócio, ao mesmo tempo em que oferece conectividade com aplicações externas por meio de APIs.

A execução de um contrato inteligente na rede Hyperledger Fabric segue um fluxo bem definido. Inicialmente, o \textit{chaincode} é desenvolvido e empacotado utilizando o \textit{Chaincode Development Kit} (CDK). Em seguida, os contratos são executados nos nós endossadores, responsáveis por validar a lógica do contrato e endossar a transação. Após o endosso, a transação é encaminhada aos nós ordenadores, que realizam a ordenação das transações e a criação de blocos conforme parâmetros predefinidos. Esses blocos têm seus hashes calculados e são posteriormente adicionados ao livro-razão distribuído por meio do mecanismo de consenso adotado pela rede, garantindo a consistência do estado do ledger \cite{AliSyed2019}.

\begin{figure}[h]
    \centering
    \includegraphics[width=1\linewidth]{images/HPFModel.jpg}
    \caption{Modelo da arquitetura Hyperledger Fabric}
    \label{fig:fig5}
\end{figure}

No que se refere aos mecanismos de segurança, o Hyperledger Fabric oferece suporte a coleções privadas de dados, permitindo que apenas participantes autorizados tenham acesso a subconjuntos específicos de informações, reforçando requisitos de confidencialidade e controle de acesso.

De forma geral, o Hyperledger Fabric tem se consolidado como uma arquitetura de livro-razão distribuído adequada para aplicações corporativas. A plataforma disponibiliza uma interface padronizada para desenvolvimento de contratos inteligentes por meio de \textit{chaincode}, compatível com múltiplas linguagens de programação, como Node.js, Go, Java e TypeScript. Além disso, fornece interfaces RESTful que facilitam a integração com aplicações existentes e o uso combinado de dados \textit{on-chain} e \textit{off-chain}.

\section{Implementação}

O sistema proposto utiliza a tecnologia blockchain Hyperledger Fabric para o registro de transações relacionadas à venda de apólices de seguro viagem, por meio da execução de contratos inteligentes. A solução envolve participantes como seguradora e corretor, sendo projetada com uma arquitetura modular e adaptável, capaz de acomodar ajustes conforme as especificidades de diferentes produtos e cenários do setor de seguros.

\subsection{Configuração da Rede Hyperledger Fabric}

A configuração da rede Hyperledger Fabric compreende a infraestrutura necessária para suportar a comunicação segura e a validação das transações entre os participantes da rede. Esse processo inclui a geração de identidades criptográficas, a definição de políticas de governança e a configuração dos canais de comunicação. As principais etapas da configuração da rede são descritas a seguir:

\renewcommand{\labelitemi}{\tiny$\bullet$}
\begin{itemize}
    \item \textit{Geração de artefatos criptográficos}: criação de certificados e chaves criptográficas para identificação e autenticação dos participantes, garantindo a segurança e a integridade das transações. Esse processo é realizado com a ferramenta \textit{cryptogen}, conforme as diretrizes do Hyperledger Fabric~\cite{Hyperledger2020};
    
    \item \textit{Definição dos arquivos de configuração}: estruturação da rede, incluindo organizações participantes, canais, políticas de consenso e configurações de peers e orderers, bem como as regras de comunicação entre os componentes;
    
    \item \textit{Criação dos contêineres Docker}: instanciação dos componentes da rede, como peers, orderers e autoridade certificadora (CA), por meio do Docker Compose;
    
    \item \textit{Inicialização dos contêineres}: ativação dos serviços definidos, tornando a rede Hyperledger Fabric operacional.
\end{itemize}
\renewcommand{\labelitemi}{\textbullet}

\subsection{Implementação do Contrato Inteligente}

O desenvolvimento do contrato inteligente abrange operações relacionadas ao ciclo de vida de uma apólice de seguro viagem, incluindo criação, atualização, consulta e cancelamento, de maneira compatível com a estrutura requerida para o armazenamento e a manipulação dos dados em uma blockchain.

A classe \texttt{Eligibility} é responsável pela validação das operações associadas às apólices, garantindo a conformidade com as regras de negócio estabelecidas. Os principais métodos dessa classe incluem:

\renewcommand{\labelitemi}{\tiny$\bullet$}
\begin{itemize}
    \item \textit{Validate policy creation}: valida a criação de uma apólice, verificando critérios como datas da viagem, destino e número de passageiros;
    \item \textit{Validate policy update}: valida a atualização da apólice, verificando a consistência do novo estado;
    \item \textit{Validate policy cancellation}: valida o cancelamento, assegurando que a apólice esteja ativa e elegível para essa operação.
\end{itemize}
\renewcommand{\labelitemi}{\textbullet}

A classe \texttt{Policy} implementa o contrato inteligente responsável pelo gerenciamento das apólices de seguro, disponibilizando métodos para a manipulação e consulta dos registros persistidos na blockchain, conforme detalhado a seguir:

\renewcommand{\labelitemi}{\tiny$\bullet$}
\begin{itemize}
    \item \textit{Init}: inicializa o \textit{chaincode};
    \item \textit{Create policy}: cria uma nova apólice a partir de uma estrutura JSON;
    \item \textit{Query policy}: consulta uma apólice pelo identificador único;
    \item \textit{Query all policies}: lista os identificadores de todas as apólices registradas;
    \item \textit{Query by holder document id}: consulta apólices associadas ao documento do titular;
    \item \textit{Update policy status}: atualiza o estado da apólice;
    \item \textit{Cancel policy}: realiza o cancelamento da apólice e inicia o processamento do reembolso;
    \item \textit{Calculate refund}: calcula o valor do reembolso com base na data de cancelamento e no prêmio contratado.
\end{itemize}
\renewcommand{\labelitemi}{\textbullet}

As transações associadas à contratação do seguro e à aplicação das regras de negócio — tais como validação de datas, verificação do destino e conferência do número de passageiros — foram implementadas em linguagem Java, de modo a garantir a compatibilidade com a plataforma Hyperledger Fabric e a eficiência na execução dos contratos inteligentes.

As classes empregadas para a representação e a manipulação dos dados das apólices no âmbito do contrato inteligente são apresentadas e descritas a seguir:

\renewcommand{\labelitemi}{\tiny$\bullet$}
\begin{itemize}
    \item \texttt{InsuranceProduct.java}: representa o produto de seguro, incluindo informações de cobertura e valor segurado;
    \item \texttt{TravelInfo.java}: armazena informações da viagem, como destino, datas de início e término e número de passageiros;
    \item \texttt{Refund.java}: representa os dados associados ao reembolso, incluindo identificador da apólice, titular, valor e status;
    \item \texttt{Policy.java}: modela a apólice de seguro, reunindo informações do titular, produto, passageiros e dados da viagem;
    \item \texttt{Passenger.java}: representa um passageiro vinculado à apólice, contendo dados pessoais e informações de identificação.
\end{itemize}
\renewcommand{\labelitemi}{\textbullet}

Essas classes são utilizadas de forma integrada para permitir a criação, validação, atualização e cancelamento de apólices, bem como a serialização e desserialização dos dados para armazenamento e recuperação segura no livro-razão distribuído da blockchain.


\subsection{Interface para Integração}

A integração com os contratos inteligentes na blockchain para registrar informações como a apólice de seguro, contendo a lógica de negócios para criar, consultar e tomar decisões para registro das transações baseadas no estado do contrato, foi feita a partir de uma aplicação Spring Boot que fornece uma API REST para gerenciar seguros de viagem.

O desenvolvimento da API REST considera, como apresentado a seguir, as principais classes com suas respectivas responsabilidades.
\renewcommand{\labelitemi}{\tiny$\bullet$}
\begin{itemize}
    \item PolicyController.java: controlador REST que expõe endpoints para gerenciar apólices de seguro;
    \item PolicyService.java: serviço que contém a lógica de negócios para gerenciar apólices de seguro;
    \item WalletService.java: serviço que gerencia a carteira de identidades do Hyperledger Fabric.
\end{itemize}
\renewcommand{\labelitemi}{\textbullet}

A classe PolicyController é um controlador REST que expõe endpoints para gerenciar apólices de seguro. Ela utiliza o serviço PolicyService, de acordo com os endpoints a seguir, para realizar operações de criação, consulta, atualização e cancelamento de apólices.

\renewcommand{\labelitemi}{\tiny$\bullet$}
\begin{itemize}
    \item POST /api/policies/create: cria apólice de seguro;
    \item GET /api/policies/{policyId}: consulta uma apólice de seguro pelo ID;
    \item GET /api/policies/findAll: consulta todas as apólices de seguro;
    \item GET /api/policies/holder/{documentId}: consulta apólices de seguro pelo documento do titular;
    \item PUT /api/policies/{policyId}/status: atualiza o status de uma apólice de seguro;
    \item POST /api/policies/cancel/{policyId}: cancela uma apólice de seguro e processa o reembolso.
\end{itemize}
\renewcommand{\labelitemi}{\textbullet}

A classe PolicyService contém a lógica de negócios para gerenciar apólices de seguro, de acordo com os métodos apresentados a seguir. É responsável por criar, consultar, atualizar e cancelar apólices, além de processar reembolsos.

\renewcommand{\labelitemi}{\tiny$\bullet$}
\begin{itemize}
    \item Create policy: cria apólice de seguro;
    \item Query policy: consulta uma apólice de seguro pelo ID;
    \item Query all policies: consulta todas as apólices de seguro;
    \item Query by holder documentId: consulta apólices de seguro pelo documento do titular;
    \item Update policy status: atualiza o status de uma apólice de seguro;
    \item Cancel policy: cancela uma apólice de seguro e processa o reembolso.
\end{itemize}
\renewcommand{\labelitemi}{\textbullet}

A classe WalletService é responsável por gerenciar a carteira de identidades do Hyperledger Fabric. Ela lida com operações relacionadas à autenticação e autorização na rede Hyperledger Fabric, incluindo a gestão de credenciais e a interação com a rede blockchain. A classe WalletService inclui métodos como:

\renewcommand{\labelitemi}{\tiny$\bullet$}
\begin{itemize}
    \item Load credentials: carrega as credenciais de um usuário específico;
    \item Store credentials: armazena as credenciais de um usuário específico;
    \item Authenticate: autentica um usuário na rede Hyperledger Fabric;
    \item Authorize: autoriza um usuário a acessar um recurso específico na rede Hyperledger Fabric.
\end{itemize}
\renewcommand{\labelitemi}{\textbullet}


\subsection{Frontend para interação de usuários}

O desenvolvimento da aplicação web teve como objetivo permitir a interação de usuários com uma rede blockchain para gestão de apólices de seguro‑viagem. Implementada com a biblioteca React, oferece criação, consulta de status e cancelamento de apólices, garantindo maior segurança, rastreabilidade e transparência nas operações.

A comunicação entre o frontend e a blockchain ocorre por meio de uma API backend, que intermedeia as requisições, converte os dados para formatos compatíveis com a rede e assegura a correta execução das transações nos contratos inteligentes. A Figura \ref{fig:fig6} apresenta a interface para criação de apólices e os menus para consulta e cancelamento.

\begin{figure}[h]
    \centering
    \includegraphics[width=1\linewidth]{images/frontend.jpg}
    \caption{Frontend para interação de usuários}
    \label{fig:fig6}
\end{figure}

Os procedimentos para construir e executar o frontend, apresentado na Figura \ref{fig:fig6}, buscaram assegurar a integridade, a rastreabilidade e a confiabilidade dos dados manipulados no sistema com o uso das melhores praticas.

\subsection{Ambiente de execução}
%2.3.6	Resultado da Implementação da Prova de Conceito

A Figura \ref{fig:fig7} apresenta os containers em execução, cada um implementando um componente do sistema de vendas de apólices de seguro.

\renewcommand{\labelitemi}{\tiny$\bullet$}
\begin{itemize}
    \item dev-insurance-sales-api: executa a API de vendas de seguros, responsável por receber e processar requisições de compra de apólices;
    \item dev-peer0.org2.example.com-insurance-chaincode: peer da organização 2, no qual é instanciado e executado o chaincode de seguro, participando da validação e do endosso das transações;
    \item dev-peer0.org1.example.com-insurance-chaincode: peer da organização 1, também responsável pela execução do chaincode de seguro e pela validação e endosso das transações;
    \item cli: disponibiliza as ferramentas de linha de comando do Hyperledger Fabric para gerenciamento, configuração e interação com a rede;
    \item peer0.org2.example.com: peer da organização 2, que mantém cópia do ledger e participa do protocolo de consenso;
    \item peer0.org1.example.com: peer da organização 1, que igualmente mantém cópia do ledger e integra o processo de consenso;
    \item orderer.example.com: executa o serviço de ordenação do Hyperledger Fabric, ordenando transações, agrupando-as em blocos e submetendo-as ao ledger;
    \item couchdb0 e couchdb1: executam instâncias CouchDB usadas como bancos de dados de estado dos peers, armazenando o estado global e viabilizando consultas eficientes e complexas;
    \item ca\_org2, ca\_org1 e ca\_orderer: executam as Autoridades Certificadoras das organizações 1, 2 e do serviço de ordenação, responsáveis por emitir, gerenciar e revogar certificados de identidade e chaves criptográficas.
\end{itemize}
\renewcommand{\labelitemi}{\textbullet}

\begin{figure}[h]
    \centering
    \includegraphics[width=1\linewidth]{images/containers.png}
    \caption{Containers implementados para o ambiente de execução}
    \label{fig:fig7}
\end{figure}

Esses containers viabilizam a operação e manutenção da rede blockchain, garantindo o processamento seguro, confiável e eficiente das transações de apólices de seguro, em conformidade com requisitos de integridade, rastreabilidade, disponibilidade e desempenho de um sistema distribuído baseado em razão distribuída.

\section{Resultados}
Nesta seção, são apresentados e discutidos os resultados dos testes de validação da segurança nas transações de venda de apólices de seguros e da eficiência do sistema, com ênfase em sua estabilidade frente ao aumento do volume de transações. Os resultados foram analisados para verificar a eficácia dos mecanismos de controle de fraude por detecção de duplicação de transações, confirmando que o contrato inteligente identifica e rejeita transações redundantes. Também foram avaliados fatores de desempenho, como a taxa de processamento em transações por segundo.

Cada cenário de teste é descrito para explicitar os desafios enfrentados e as soluções adotadas na criação de apólices. A análise busca identificar como variáveis como volume de transações e mecanismos de controle influenciam o desempenho e a segurança do sistema. Os resultados são discutidos com foco nas transações de venda de apólices de seguros e na avaliação da viabilidade técnica da arquitetura proposta.

\subsection{Análise de Segurança}
A integridade da apólice de seguro é requisito fundamental para garantir a confiabilidade, a segurança e a imutabilidade das informações na blockchain. Para avaliar a capacidade do sistema em protegê-la contra acessos não autorizados e adulterações, foram definidos três cenários de teste que analisam diferentes aspectos de segurança e proteção de dados na rede Hyperledger Fabric.

Os cenários da Tabela \ref{tab:tab1} verificam se as políticas de autenticação e autorização estão corretamente implementadas e se o sistema mantém a consistência dos dados na blockchain, impedindo modificações indevidas. Assim, avalia-se a capacidade do sistema em salvaguardar a apólice ao longo de todas as etapas do processo.

Os testes validaram que a aplicação preserva a segurança e a integridade das apólices, prevenindo acessos e alterações não autorizados. Confirmou-se que os contratos armazenados na blockchain permanecem confiáveis, imutáveis e acessíveis apenas a participantes autorizados.

Os resultados demonstram a eficácia da aplicação no controle das apólices: no Cenário 1, as 50 tentativas de modificação não autorizada foram bloqueadas; no Cenário 2, todas as transações legítimas foram processadas corretamente; e no Cenário 3, 50 tentativas de acesso indevido foram impedidas e registradas como eventos suspeitos, comprovando a integridade e a segurança do sistema.

\begin{sidewaystable}%
\centering
\caption{Cenários de teste para segurança, controle de acesso e integridade dos dados da apólice de seguro viagem.\label{tab:tab1}}%
\small
\begin{tabular*}{\textheight}{@{\extracolsep\fill}
p{3.1cm}   % Cenário
p{3.7cm}   % Objetivo
p{4.4cm}   % Procedimento
p{4.4cm}   % Critérios de Sucesso
p{4.5cm}   % Resultados
@{\extracolsep\fill}}%
\toprule
\textbf{Cenário} &
\textbf{Objetivo} &
\textbf{Procedimento} &
\textbf{Critérios de Sucesso} &
\textbf{Resultados} \\
\midrule

\textbf{Cenário 1:} Tentativa de modificação não autorizada da apólice
&
Garantir que apenas usuários autorizados possam modificar os dados da apólice.
&
Criar uma apólice com usuário autorizado; tentar alterar seus dados com usuário não autorizado; monitorar respostas da API e logs da blockchain.
&
O sistema deve bloquear a modificação, registrar a tentativa e manter a versão original da apólice na blockchain.
&
A apólice permaneceu inalterada após tentativas não autorizadas, demonstrando que a arquitetura restringe corretamente modificações apenas a usuários autorizados.
\\

\textbf{Cenário 2:} Verificação da integridade dos dados da apólice
&
Assegurar que os dados da apólice não sejam alterados fora do fluxo transacional autorizado.
&
Criar a apólice e registrar seu hash na blockchain; realizar operações normais; comparar o hash original com o hash atual.
&
O hash da apólice deve permanecer inalterado, exceto em modificações legítimas.
&
Todas as transações executadas preservaram a integridade dos dados, mantendo consistência entre os hashes armazenados e os registros da apólice.
\\

\textbf{Cenário 3:} Acesso indevido à apólice de seguro
&
Testar a robustez do sistema contra tentativas de acesso não autorizado a dados sensíveis.
&
Criar a apólice com permissões restritas; simular acessos não autorizados via API e com credenciais inválidas; analisar respostas e logs.
&
O sistema deve bloquear o acesso não autorizado e registrar eventos suspeitos.
&
As tentativas de acesso indevido foram bloqueadas com sucesso e devidamente registradas, evidenciando a robustez do controle de acesso.
\\

\bottomrule
\end{tabular*}
\end{sidewaystable}

\subsection{Análise de Desempenho}

Esta subseção apresenta uma avaliação do desempenho do processo de comercialização de apólices de seguro, em termos de tempo de resposta, para a arquitetura proposta. O objetivo é identificar de que forma o número de transações por segundo e os volumes de dados influenciam o comportamento do sistema em condições operacionais reais.

A análise de desempenho tem como finalidade mensurar a eficiência da aplicação voltada à venda de apólices de seguro, considerando tanto o tempo de resposta quanto a escalabilidade do sistema sob diferentes perfis de carga de trabalho. Os testes descritos na Tabela \ref{tab:tab2} foram concebidos para quantificar a capacidade do sistema de processar transações em alta taxa e sob grandes volumes de dados, assegurando um funcionamento adequado em cenários de uso realista.

Os três cenários de teste permitiram avaliar de maneira sistemática a eficiência e a escalabilidade do sistema em distintas condições operacionais, verificando se a aplicação descentralizada é capaz de suportar altas cargas de transações, grandes volumes de dados e executar consultas de forma eficiente.

Os testes de desempenho foram realizados em ambiente controlado para cada um dos 3 cenários, e os resultados validaram a eficiência da aplicação na venda de apólice de seguros para a aplicação proposta.

Para o Cenário 1, em que o teste para medir o tempo médio de resposta para o processamento de uma única transação de venda de apólice e garantir que o tempo esteja dentro dos limites aceitáveis, foi realizado a criação de 1000 apólices individuais utilizando um payload com 332 bytes para cada requisição. O tempo médio total para criação de uma apólice foi de 2140 ms. O tempo máximo apresentado para uma transação foi de 4939 ms, não atendendo ao critério de sucesso previamente determinado, possivelmente devido a variações na conexão com a blockchain para a primeira requisição realizada. A mediana do tempo total ficou muito próximo da média, em 2124 ms, indicando uma distribuição equilibrada dos tempos de processamento. O tempo mínimo registrado foi de 2078 ms. Para as 1000 requisições realizadas foi alcançado uma taxa de sucesso de 99,9\% mostrando estabilidade na execução das transações.

Para o Cenário 2, com a finalidade de avaliar o desempenho do sistema sob carga crescente, medindo o impacto no tempo de resposta à medida que o número de transações simultâneas aumenta, diferentes níveis de carga foi simulado, iniciando com 100 transações por segundo e aumentando progressivamente para 500, 1000 e 2000 transações por segundo. A Figura \ref{fig:fig8}, apresenta o resultado para os testes de escalabilidade.


\begin{sidewaystable}%
\centering
\caption{Cenários de Teste para Avaliação do Desempenho da Aplicação.\label{tab:tab2}}%
\small
\begin{tabular*}{\textheight}{@{\extracolsep\fill}
p{3.1cm}   % Cenário
p{4.7cm}   % Objetivo
p{6.9cm}   % Procedimento
p{6.9cm}   % Critérios de Sucesso
@{\extracolsep\fill}}%
\toprule
\textbf{Cenário} &
\textbf{Objetivo} &
\textbf{Procedimento} &
\textbf{Critérios de Sucesso} \\
\midrule

\textbf{Cenário 1:} Teste de Tempo de Resposta por Transação
&
Medir o tempo médio de resposta para o processamento de uma única transação de venda de apólice e garantir que o tempo esteja dentro dos limites aceitáveis.
&
Simular a criação de 1000 apólices individuais, uma por vez, registrando o tempo necessário para cada operação.
Monitorar o tempo médio de resposta do contrato inteligente na blockchain.
Avaliar se o tempo de resposta se mantém estável à medida que o volume de transações aumenta.
&
O tempo médio de resposta por transação deve estar dentro dos limites aceitáveis (menor a 3 segundos, de acordo com experiência empírica) para garantir a fluidez da aplicação, sem atrasos significativos.
\\

\textbf{Cenário 2:} Teste de Escalabilidade com Carga Aumentada
&
Avaliar o desempenho do sistema sob carga crescente, medindo o impacto no tempo de resposta à medida que o número de transações simultâneas aumenta.
&
Simular diferentes níveis de carga, iniciando com 100 transações por segundo e aumentando progressivamente para 500, 1000 e 2000 transações por segundo.
Monitorar a latência, uso de CPU, memória e tempo de commit das transações na blockchain.
Identificar o ponto de saturação do sistema e se há degradação significativa no desempenho.
&
O sistema deve manter o número de TPS sem degradação. Caso haja degradação, deve ser identificada a capacidade máxima suportada antes de comprometer a usabilidade.
\\

\textbf{Cenário 3:} Teste de Desempenho com Grandes Volumes de Dados
&
Avaliar o impacto da manipulação de grandes volumes de dados no tempo de resposta e na eficiência do sistema.
&
Criar um banco de dados simulado contendo mais de 100 mil apólices.
Executar consultas complexas, como busca por titular e listagem de apólices.
Medir o tempo de resposta para cada consulta e verificar a eficiência da busca em grandes volumes de dados.
&
Baseado em experiencia empírica, o sistema deve responder em menos de 3 segundos às consultas, garantindo a escalabilidade e a eficiência da arquitetura para processar grandes quantidades de informações sem comprometer a performance.
\\

\bottomrule
\end{tabular*}
\end{sidewaystable}

A primeira simulação, com 100 apólices, apresentou tempo médio total de 2280 ms para a criação de uma apólice, valor muito próximo ao observado no Cenário 1, de 2140 ms. A mediana foi de 2233 ms, próxima da média, o que indica consistência no processo de venda de apólices. O tempo mínimo observado foi de 973 ms e o máximo de 3636 ms, resultando em uma amplitude de 2663 ms, atribuída principalmente ao tempo de conexão e resposta da blockchain, evidenciando maior oscilação à medida que se eleva o volume de transações.

A estabilidade das transações torna-se mais perceptível com o aumento do volume para 500, 1000 e 2000 transações por segundo. Observa-se, entretanto, uma taxa de sucesso de 99,8\% nas simulações com 500 e 1000 transações por segundo e de 99,6\% na simulação com 2000 transações por segundo, sendo esta redução associada a falhas na conexão com a blockchain. 

O maior throughput foi obtido na simulação com 500 apólices, com 27,8 transações processadas por segundo. Contudo, a simulação com 100 apólices apresentou throughput de 25,0 transações por segundo com taxa de sucesso de 100\%, o que demonstra maior estabilidade operacional no processo de venda de apólices.

\begin{figure}[h]
    \centering
    \includegraphics[width=1\linewidth]{images/escalabilidade.jpg}
    \caption{Resultado do Teste de Escalabilidade}
    \label{fig:fig8}
\end{figure}

Para o Cenário 3, com o objetivo de avaliar o impacto da manipulação de grandes volumes de dados sobre o tempo de resposta e a eficiência do sistema, foi realizada a carga de 135.800 apólices. A Figura \ref{fig:fig9} apresenta a quantidade de transações efetuadas na fase de preparação dos testes.

\begin{figure}[h]
    \centering
    \includegraphics[width=1\linewidth]{images/tps.png}
    \caption{Quantidade de Transações Realizadas}
    \label{fig:fig9}
\end{figure}

Foi realizada uma consulta por identificador (ID) de apólice, apresentando um tempo de resposta satisfatório de 91,87 ms. A Figura \ref{fig:fig10} ilustra a consulta executada por meio de uma API REST, evidenciando um resultado considerado satisfatório.

\begin{figure}[h]
    \centering
    \includegraphics[width=\linewidth]{images/ConsultaPolicybyID.png}
    \caption{Resultado de Consulta de Apólice pelo ID}
    \label{fig:fig10}
\end{figure}

Posteriormente, foi efetuada uma consulta utilizando como parâmetro de filtragem o documento de identificação do titular da apólice, cujo tempo de resposta excedeu 30 segundos, resultando em falha por timeout. Esse comportamento indica que o processamento da requisição demandou um intervalo superior ao limite esperado pela aplicação cliente, deixando de atender aos critérios de desempenho previamente estabelecidos.

A Figura \ref{fig:fig11}, a seguir, apresenta o resultado da consulta realizada, com o erro gerado após superar o limite de tempo configurado na aplicação cliente.

\begin{figure}[h]
    \centering
    \includegraphics[width=1\linewidth]{images/SearchByHolderId.png}
    \caption{Resultado para busca de apólice com o Id do titular}
    \label{fig:fig11}
\end{figure}

No Cenário 3, foi realizada uma requisição por meio de API REST para a obtenção da lista de identificadores (IDs) de apólices existentes. O tempo de resposta excedeu 30 segundos, ocasionando a ocorrência de um erro de timeout na aplicação e, consequentemente, o não atendimento aos critérios de desempenho estabelecidos. A Figura \ref{fig:fig12} apresenta a requisição efetuada, bem como a mensagem de erro retornada na resposta à consulta.

\begin{figure}[h]
    \centering
    \includegraphics[width=1\linewidth]{images/SearchAll.png}
    \caption{Resultado de consulta para todas as apólices}
    \label{fig:fig12}
\end{figure}

O sistema não apresentou desempenho satisfatório na execução da consulta parametrizada pelo identificador (Id) da apólice. De modo análogo, nas consultas em que se empregou, como critério de filtragem, o documento do titular da apólice ou a requisição para que a aplicação gerasse a lista contendo o Id de todas as apólices emitidas, o sistema evidenciou não conformidade em relação aos critérios previamente estabelecidos. Diante disso, torna-se necessária a revisão do modelo, de forma a possibilitar a realização das consultas sem prejuízos operacionais, mesmo em cenários de grande volume de dados, assegurando a manutenção dos níveis de desempenho e eficiência do sistema.

\subsection{Discussão}
Esta subseção de discussão foi estruturada em duas dimensões principais: segurança e desempenho. Na dimensão de segurança, examina-se os mecanismos de proteção adotados, com ênfase no controle de fraude por meio da garantia de integridade da informação, bem como na capacidade de resiliência da aplicação diante de cenários de comprometimento. Na dimensão de desempenho, analisa-se a eficiência do processo de comercialização de apólices de seguro, especialmente no que se refere ao tempo de resposta da aplicação proposta.


\subsubsection{Segurança}
A integridade da apólice de seguros constitui requisito fundamental para assegurar a confiabilidade, a segurança e a imutabilidade das informações registradas em blockchain. No contexto da solução baseada na rede Hyperledger Fabric, foram desenvolvidos três cenários de teste com o objetivo de avaliar a capacidade do sistema de proteger a apólice contra acessos não autorizados e modificações indevidas. Esses testes foram concebidos para validar a implementação de políticas de autenticação e autorização, bem como para verificar a consistência dos dados armazenados na blockchain.

Os cenários apresentados no Table \ref{tab:tab1} descrevem de forma detalhada os procedimentos executados e os critérios de sucesso estabelecidos para cada um deles. Os testes foram estruturados de modo a abranger diferentes dimensões de segurança e proteção de dados na rede blockchain.

A execução dos testes possibilitou verificar se a aplicação mantém a segurança e a integridade das apólices de seguro, prevenindo acessos indevidos e alterações não autorizadas. A abordagem adotada assegurou que os contratos armazenados na blockchain permanecessem confiáveis, imutáveis e acessíveis apenas aos participantes devidamente autorizados no sistema.

\subsubsection{Desempenho}
A eficiência da aplicação na venda de apólices de seguro é um fator essencial para garantir a viabilidade da aplicação proposta, especialmente em ambientes que demandam alta escalabilidade e confiabilidade. Dessa forma, a análise de desempenho visa medir a capacidade da aplicação descentralizada baseada em blockchain de processar transações de forma eficiente e com tempos de resposta aceitáveis sob diferentes cenários de carga e volume de dados.

Para avaliar essa eficiência, foram definidos três cenários principais de teste, conforme detalhado na Table \ref{tab:tab2}. Esses cenários foram projetados para analisar o tempo de resposta por transação, a escalabilidade do sistema sob cargas crescentes e a capacidade da aplicação de lidar com grandes volumes de dados sem comprometer a performance.

O Cenário 1, teve como propósito a medição do tempo médio de resposta para o processamento de uma transação individual de venda de apólice, verificando se o sistema atende aos limites aceitáveis para um funcionamento eficiente. Para isso, foram criadas 1000 apólices de forma sequencial, registrando o tempo necessário para cada operação.

O Cenário 2, teve como foco a avaliação do impacto do aumento do número de transações simultâneas no tempo de resposta da aplicação. Foram simuladas cargas progressivas de 100, 500, 1000 e 2000 transações por segundo, monitorando a latência, o uso de CPU e memória, bem como o tempo de commit das transações na blockchain.

O Cenário 3, avaliou o impacto da manipulação de grandes volumes de dados na eficiência da aplicação, simulando um banco de dados com mais de 100 mil apólices. Foram realizadas consultas por ID da apólice, por titular e a listagem completa das apólices cadastradas.

Os testes evidenciaram pontos de atenção relacionados ao desempenho em consultas sobre grandes conjuntos de dados, que exigem ajustes para garantir uma performance consistente e responsiva em cenários de alta densidade de informações. Esses pontos, embora não comprometam o funcionamento principal da aplicação, são relevantes para sua adoção em ambientes de produção com alta demanda analítica.



\section{Limitações e Trabalhos Futuros}

\textbf{Limitações:} Apesar das contribuições deste estudo, algumas limitações devem ser consideradas. Em primeiro lugar, a prototipação da prova de conceito foi conduzida em um ambiente controlado, utilizando uma infraestrutura baseada em contêineres com Docker, o que pode divergir de cenários reais de produção, nos quais requisitos de escalabilidade e resiliência da rede tendem a impor desafios adicionais. Ademais, a aplicação proposta foi concebida para um caso específico de seguro-viagem, o que pode demandar adaptações para sua utilização em outros ramos de seguros, bem como para o atendimento a diferentes marcos regulatórios em distintas jurisdições. 

Outra limitação refere-se à complexidade inerente à tecnologia blockchain, que pode requerer investimentos significativos em capacitação técnica e em infraestrutura para viabilizar sua adoção em larga escala pelas organizações do setor. Por fim, aspectos relacionados à governança da rede e à integração com sistemas legados não foram explorados em profundidade neste trabalho, embora representem desafios críticos para a implementação prática da solução em ambientes corporativos reais.

\textbf{Trabalhos futuros:} Diante das limitações identificadas, diversas direções para investigações futuras podem ser delineadas. Inicialmente, a realização de experimentos em ambientes de produção ou em redes blockchain permissionadas de larga escala permitiria avaliar, de forma mais robusta, a escalabilidade e o desempenho da solução sob condições reais de operação. Adicionalmente, a adaptação da aplicação para contemplar distintos tipos de seguros, levando em consideração regulamentações específicas de diferentes jurisdições, poderia ampliar sua aplicabilidade e relevância no setor securitário.

Investigações subsequentes podem concentrar-se na proposição de estratégias de otimização do desempenho do contrato inteligente, com foco na redução de custos computacionais e na mitigação da latência das transações na rede. A incorporação de técnicas avançadas, como o emprego de inteligência artificial para análise de risco e modelagem preditiva, configura-se como uma possibilidade promissora para agregar valor à solução, aumentando sua eficiência operacional e sua capacidade de suporte à tomada de decisão.

Outra vertente de evolução envolve o aperfeiçoamento dos mecanismos de governança da rede blockchain, por meio da análise e adoção de diferentes mecanismos de consenso e modelos de gestão descentralizada, com o objetivo de definir, de maneira transparente e auditável, regras de participação e processos decisórios para o registro de transações entre múltiplos atores do ecossistema. A integração da aplicação com plataformas legadas das seguradoras também se apresenta como um campo de estudo relevante, possibilitando uma transição mais gradual e interoperável para o uso de blockchain, sem descontinuar ou comprometer fluxos de trabalho já consolidados.

No que tange à segurança em ambientes blockchain, esta constitui um requisito fundamental para assegurar a integridade e a autenticidade das transações, sobretudo em sistemas que manipulam informações sensíveis, como dados de apólices de seguro e informações pessoais de clientes. Uma potencial linha de aprimoramento consiste na implementação de mecanismos de controle de acesso mais sofisticados, incluindo a integração de sistemas de autenticação multifatorial (MFA) e a adoção de algoritmos criptográficos de última geração, de modo a proporcionar maior resiliência frente a ataques de força bruta e outras vulnerabilidades. Ademais, a ampliação do uso de criptografia de ponta a ponta (end-to-end) poderia elevar o nível de proteção dos dados em trânsito, aspecto crítico dada a natureza sensível das informações tratadas.

Por fim, avanços relacionados à realização de testes de intrusão (penetration tests) específicos para contratos inteligentes e à implementação de soluções de monitoramento contínuo e em tempo real, voltadas à detecção precoce de atividades anômalas e tentativas de fraude, têm o potencial de reforçar substancialmente a resiliência global do sistema.

Em síntese, pesquisas orientadas à melhoria do desempenho e ao fortalecimento dos mecanismos de segurança, em conjunto com o aperfeiçoamento contínuo dos controles de acesso, dos processos de auditoria e das capacidades de monitoramento, representam um avanço significativo para a consolidação da confiança, da integridade e da conformidade da solução blockchain em ambientes do setor de seguros.


\section{Conclusão}

Os objetivos estabelecidos no início desta dissertação foram integralmente alcançados. Em primeiro lugar, foi conduzida uma revisão sistemática da literatura que evidenciou o estado da arte referente a aplicações descentralizadas baseadas em blockchain para a comercialização de apólices de seguros. Essa revisão forneceu uma base conceitual e tecnológica robusta para o desenvolvimento da aplicação descentralizada proposta, destacando as principais tecnologias e frameworks disponíveis na área. Na sequência, foi especificada uma arquitetura destinada ao desenvolvimento de um processo seguro para a venda de apólices de seguros, utilizando a tecnologia blockchain como infraestrutura para garantir descentralização, persistência, anonimato e auditabilidade das transações.

A prototipagem da prova de conceito evidenciou a viabilidade do uso de blockchain, especificamente da plataforma Hyperledger Fabric, para a comercialização de apólices de seguro. A adoção de uma infraestrutura baseada em containers assegurou um ambiente controlado, isolado e reprodutível, enquanto a configuração detalhada da rede garantiu a autenticidade, integridade e segurança das transações realizadas. O contrato inteligente desenvolvido permitiu a automatização de operações essenciais do domínio de seguros, promovendo transparência e confiabilidade a todo o processo transacional. Adicionalmente, a integração com uma API REST possibilitou a comunicação eficiente entre os usuários e a rede blockchain, resultando em um sistema com potencial de robustez e escalabilidade.

Os testes funcionais executados validaram a correta implementação e execução das principais operações do sistema, incluindo a criação, consulta, atualização e cancelamento de apólices, bem como o cálculo de reembolsos. Os resultados demonstraram que as regras de negócio foram adequadamente aplicadas e que o desempenho do sistema esteve em conformidade com os requisitos e expectativas estabelecidos, assegurando a integridade, consistência e confiabilidade das transações. Dessa forma, os achados deste trabalho corroboram o potencial da tecnologia blockchain para promover inovações significativas no setor de seguros, contribuindo para processos mais eficientes, seguros e confiáveis nas transações envolvendo apólices.

\bibliographystyle{plain}
\bibliography{references}

\end{document}
