% \documentclass[AMA,Times1COL]{WileyNJDv5}
\documentclass[LATO2COL,table]{WileyNJDv5}

\articletype{Original Research}%

\received{Date Month Year}
\revised{Date Month Year}
\accepted{Date Month Year}
\journal{Journal}
\volume{00}
\copyyear{2025}
\startpage{1}

\raggedbottom
\usepackage{hyperref}
\usepackage{unicode-math}
\usepackage{fontspec}
\usepackage{graphicx}
\usepackage{enumitem}
\usepackage{xcolor}
\usepackage{booktabs}
\setmainfont[
  BoldFont={Arial-BoldMT.ttf},
  ItalicFont={Arial-ItalicMT.ttf},
  BoldItalicFont={Arial-BoldItalicMT.ttf},
  Path=./Fonts/Arial/
]{Arial.ttf}

\setmathfont{Latin Modern Math}
\begin{document}

\title{A decentralized application based on blockchain for insurance policy sales}

\author[1]{Alex Prado}
\author[1]{Eduardo Ueda}
\author[1]{Anderson Silva}
\author[2]{Vilc Rufino}

\authormark{PRADO \textsc{et al.}}

\titlemark{A decentralized application based on blockchain for insurance policy sales}

\address[1]{\orgname{Instituto de Pesquisas Tecnológicas do Estado de São Paulo}, \orgaddress{\state{São Paulo}, \country{Brazil}}}

\address[2]{\orgname{Marinha do Brasil}, \orgaddress{\state{São Paulo}, \country{Brazil}}}

\corres{Alex Prado, Instituto de Pesquisas Tecnológicas do Estado de São Paulo \email{alex.prado@ensino.ipt.br}}

%\fundingInfo{Text}
%\JELinfo{ejlje}

\abstract[Abstract]{The insurance industry faces challenges such as fraud detection and prevention, as well as protecting against malicious policy participants. As a financial tool, an insurance policy serves as a guarantee against financial losses resulting from specific events, such as accidents, property damage, illnesses, or death. To meet these challenges, insurers must adapt to modern technologies such as blockchain to combat fraud and optimize policy management. The adoption of blockchain and smart contracts is accelerating, with a forecasted commercial value of approximately USD 3.1 trillion by 2030. Smart contracts can find potential application scenarios in the digital economy and smart industries and are integrated into major blockchain-based development platforms. However, there are limitations to the practical use of these methods for the following reasons: low transactions per second (TPS), block size issues, chain size growth, and electronic signature sizes. Thus, this dissertation aims to propose a solution based on private blockchain using smart contracts to support automatic decision-making in the insurance policy sales process, considering the contracts negotiated between insurers and policyholders. The research methodology consists of two main stages: a systematic literature review to gain a broad understanding of the problem and experimental research with a quantitative approach to implement and evaluate the proposed solution. The expected contributions were to add to the scientific literature applications of decision-making with smart contracts in the blockchain network in the insurance industry and to present an implementation proposal for a blockchain-based system applied to insurance policy sales operations with support for automatic decision-making.}

\keywords{Blockchain, Smart contracts, Insurance Policy, Decision-making, Insurance Sales}

\jnlcitation{\cname{%
\author{Alex P.},
\author{Eduardo E},
\author{Anderson S}, and
\author{Vilc R}}.
\ctitle{On simplifying ‘incremental remap’-based transport schemes.} \cjournal{\it J Comput Phys.} \cvol{2021;00(00):1--18}.}

\maketitle

\renewcommand\thefootnote{}
\footnotemark\footnotetext{\textbf{Abbreviations:} TPS,  transactions per second; APC, antigen-presenting cells; IRF, interferon regulatory factor.}

\renewcommand\thefootnote{\fnsymbol{footnote}}
\setcounter{footnote}{1}

\section{Introdução}\label{sec1}

O método convencional de documentar informações do seguro contratado é frequentemente inseguro, ineficiente e suscetível de manipulação. No entanto, uma vez documentados, os dados devem permanecer inalteráveis e resistentes à manipulação. Esta característica é essencial no contexto do seguro, onde a exatidão e integridade dos registros desempenham um papel fundamental na resolução de litígios e na garantia da confiança entre as partes interessadas (SODHI; DAS; LOGANATHAN, 2024).

O sistema brasileiro de consulta a registros de seguros, destinado a usuários na qualidade de segurados, foi instituído com o objetivo de viabilizar a verificação da existência de certificado ou apólice de seguro emitidos por empresas autorizadas pela Superintendência de Seguros Privados (SUSEP), vinculados ao Cadastro de Pessoa Física (CPF) do consulente. Todavia, o referido sistema não contempla informações relativas a planos e seguros de saúde, à previdência complementar aberta, a títulos de capitalização, a seguros cujo período de vigência já tenha se encerrado, a contratos firmados há menos de quatro dias úteis, bem como a determinados seguros de pessoas, como, por exemplo, seguros de vida e de acidentes pessoais (GOV.BR, 2024).

As empresas recebem uma crescente pressão para desenvolver modelos de negócio mais sustentáveis para um mundo em constante mudança, para criar valor, não apenas para os acionistas, mas também para consumidores e população afetada (BALDASSARRE et al., 2017).

Desde a primeira revolução industrial, muitos engenheiros tentam resolver problemas relacionados às operações e, ao fazê-lo, tentam melhorar a eficiência dos processos produtivos. Até 2050, a população do planeta está destinada a atingir 9,3 bilhões e diferentes desafios e questões estão aumentando a pressão sobre a indústria atual (EL HAMDI; ABOUABDELLAH; OUDANI, 2019).

Conforme observado por Amponsah; Adekoya; Weyori (2021), a indústria depende de vários processos entre as partes para iniciar, manter e encerrar diferentes transações. Consequentemente, o tempo de processamento das transações, a liquidação, o tempo de pagamento e a segurança da execução do processo são as principais preocupações.

De acordo com a proposta de Roriz e Pereira (2019), a construção de um aplicativo na blockchain, no qual cada \textit{node} é conectado a todos os outros \textit{nodes} pela rede, pode ajudar a eliminar os problemas de segurança, levando em conta as características inerentes da blockchain em conjunto com os mecanismos de consenso.

Roriz e Pereira (2019), indicam que ao fazer parte da rede blockchain, cada seguradora tem uma cópia dos dados e do código, eliminando um único ponto de falha que poderia afetar toda a rede, pois se tornariam como computadores interagindo entre si na mesma rede.

A Figura 1, a seguir, ilustra a proposta de Roriz e Pereira (2019), indicando que o sistema de uma seguradora é representado por um \textit{node} da rede blockchain. Quando é criada uma apólice de seguro, ela primeiro deve passar por todos os critérios presentes no contrato inteligente e só então a transação pode ser submetida ao blockchain. Na solução proposta, todas as seguradoras devem fazer parte da rede blockchain.

\begin{figure}[h]
    \centerline{\includegraphics[width=0.85\linewidth]{images/Figura1-ArquiteturaBlockchain.jpg}}
    \caption{Arquitetura de solução blockchain para seguradoras}
    \label{fig:fig1}
\end{figure}

Segundo Amponsah; Adekoya e Weyori (2021), em caso de emergência, o seguro é uma das assistências essenciais acessíveis às populações para neutralizar os seus custos e assisti-las. O maior desafio do setor é como detectar e proteger fraudes e impedir as más intenções dos falsos participantes. Desta forma, o impacto positivo da tecnologia blockchain foi observado pelas principais seguradoras e resseguradoras, resultando em investimento em sistemas experimentais.

Segundo indicado por Xiao et al. (2020), as seguradoras que oferecem produtos que possuem coberturas para bens como o imóvel ou o automóvel, poderiam reunir os dados de dispositivos conectados e analisá-los para um gerenciamento preciso durante o processo de sinistros e desta forma impedir as más intenções de falsos participantes.

O diagrama apresentado na Figura 2, ilustra o caso de uso apresentado por Xiao et al. (2020). Neste caso de uso estão envolvidos três tipos de participantes, sendo os motoristas de veículos, os operadores de transporte, e as companhias de seguros. A interação e a relação entre estes participantes exigem a implantação de dispositivos IoT (Internet of Things) a partir dos quais é possível realizar a coleta dos dados onde são processados pelo Sistema de Informação Geográfica para que seja possível a análise e posteriormente a gestão do processo de seguros.

\begin{figure}[h]
    \centering{\includegraphics[width=1\linewidth]{images/Figura2-TransportInsurance.png}}
    \caption{Caso de uso de aplicação de seguro de transporte}
    \label{fig:fig2}
\end{figure}

Os contratos inteligentes podem encontrar potenciais cenários de aplicação na economia digital e indústrias inteligentes, incluindo serviços financeiros, saúde e IoT, entre outros, e estão integrados às principais plataformas de desenvolvimento baseadas em blockchain, como Ethereum (ETH). O ETH é uma rede pública, um blockchain e um protocolo de código aberto, operado, governado, gerenciado e de propriedade de uma comunidade global de dezenas de milhares de desenvolvedores, operadores de nós, usuários e detentores de ETH (ETHEREUM, 2024). Já o Hyperledger Fabric que serve como base para o desenvolvimento de aplicações ou soluções com arquitetura modular permite componentes intercambiáveis, incluindo serviços de consenso e adesão, possibilitando um ambiente plug-and-play (HYPERLEDGER FOUNDATION, 2024).

O objetivo principal deste artigo é propor o uso do blockchain e contratos inteligentes na indústria de seguros na etapa de venda de uma apólice de seguros. Para isso, apresenta uma implementação de uma aplicação descentralizada baseado em blockchain aplicado em operações de vendas de apólices de seguros e contratos inteligentes para registro das transações mapeando as melhores práticas observadas e assim contribuindo para aplicação na indústria. 

\section{Trabalhos correlatos}\label{sec2}

A adoção de blockchain e contratos inteligentes pode transformar o setor de seguros, promovendo automação, transparência e redução de custos (Shetty et al., 2022; Aleksieva et al., 2020a; Xiao et al., 2020). Apesar das oportunidades, desafios como integração com sistemas legados, escalabilidade, conformidade regulatória e adaptação dos envolvidos ainda precisam ser superados (Aleksieva et al., 2020b; Bai et al., 2022). O uso de plataformas como Hyperledger Fabric e Ethereum permite flexibilidade, mas exige análise dos custos e impactos operacionais (Aleksieva et al., 2020c; Liu et al., 2019).

\subsection{Desafios e Oportunidades na Indústria de Seguros}\label{sec2.1}
O mercado de seguros está em constante transformação, exigindo das empresas adaptação às inovações tecnológicas. Shetty et al. (2022) destacam que a sobrevivência das seguradoras depende da adoção de tecnologias emergentes, como inteligência artificial (IA) e blockchain, para evitar disrupções e manter a competitividade. A blockchain, como tecnologia de banco de dados compartilhado, oferece transparência, rastreabilidade e imutabilidade, além de permitir o desenvolvimento de novos produtos e aprimorar a detecção de fraudes por meio de contratos inteligentes. Contudo, os autores ressaltam que a interface entre o mundo físico e o digital pode ser um ponto frágil, dificultando a plena confiança digital no setor.

Apesar do potencial da blockchain, Shetty et al. (2022) não aprofundam desafios como obstáculos regulatórios, integração com sistemas legados e preparação de corretores e clientes para ambientes digitalizados. Aleksieva et al. (2020b) reforçam que o uso da blockchain visa tornar a compra e venda de apólices mais eficiente, reduzindo custos e aumentando a transparência. Contratos inteligentes, especialmente os baseados em Ethereum, trouxeram automação e flexibilidade, mas os autores não discutem suficientemente desafios operacionais, como compliance, segurança, governança e custos de blockchains públicos.

Xiao et al. (2020) apontam que a blockchain resolve problemas de confiança e permite automação de processos, mas não detalham desafios na venda de apólices, como complexidade dos contratos inteligentes, adaptação regulatória e interoperabilidade. Aleksieva et al. (2020a) destacam a modernização do setor com blockchain, mas tratam superficialmente questões de escalabilidade e processamento de transações, além de não abordar como contratos inteligentes lidam com variáveis subjetivas e regulamentações locais.

No seguro agrícola, Bai et al. (2022) mostram que blockchain e IoT podem evitar manipulação de dados e reduzir erros em pagamentos de sinistros, mas não discutem a complexidade de integrar blockchain em ambientes com múltiplos participantes. Liu et al. (2019) sugerem o uso de blockchain para aprimorar o seguro-viagem, mas não detalham sua aplicação na venda de apólices. Aleksieva et al. (2020c) e Klapkiv e Kedra (2022) ressaltam que blockchains permissionadas podem resolver questões de privacidade, mas falta análise prática sobre sua aplicação na venda de apólices.


\subsection{Aplicações de Blockchain e Contratos Inteligentes na Indústria de Seguros}\label{sec2.2}
De acordo com Shetty et al. (2022), a blockchain oferece amplas oportunidades para o setor de seguros, incluindo:

\renewcommand{\labelitemi}{\tiny$\bullet$}
\begin{itemize}
    \item Seguro de viagem e de vida com pagamentos imediatos;
    \item Automação e simplificação do gerenciamento de sinistros;
    \item Jornada de sinistros mais clara em seguros de acidentes pessoais;
    \item Manutenção eficiente de registros;
    \item Automação de resseguros via contratos inteligentes;
    \item Digitalização de identidades;
    \item Seguros peer-to-peer sem intermediários
    \item Acesso em tempo real a informações sobre títulos de garantia.
\end{itemize}
\renewcommand{\labelitemi}{\textbullet}

O uso da blockchain no setor de seguros pode aumentar eficiência, transparência e satisfação do cliente, abrangendo desde seguros de viagem e vida até gerenciamento de sinistros e identidade digital (Shetty et al., 2022). No entanto, faltam discussões sobre desafios práticos, como integração com sistemas legados, implicações regulatórias e proteção de dados. Grandes seguradoras já exploram a tecnologia, mas questões econômicas e de privacidade permanecem pouco abordadas (Xiao et al., 2020). Contratos inteligentes podem automatizar sinistros, mas dependem de sistemas seguros e auditáveis (Aleksieva et al., 2020a)

O texto destaca que Aleksieva et al. (2020a) apresenta benefícios do blockchain e contratos inteligentes no processamento de sinistros, citando exemplos como o uso da tecnologia Corda pela Cat XoL Property Catastrophe Excess of Loss Reassurance e a plataforma Etherisc. No entanto, não aborda suficientemente as vantagens dessas tecnologias na venda de apólices, como automação de propostas e execução transparente de contratos, nem discute integração com sistemas legados ou interoperabilidade entre plataformas.

Aleksieva et al. (2020b) explora plataformas inovadoras como Nexus Mutual, Insureum e Savemyluggage, que oferecem produtos de seguros especializados via contratos inteligentes. Apesar de demonstrar a versatilidade do blockchain, o texto não aprofunda como essas soluções podem transformar a venda de apólices tradicionais, nem analisa custos de implementação ou impactos na negociação com corretores e clientes.

Bai et al. (2022) ressalta a integração de IoT e blockchain, destacando benefícios como detecção de fraudes e automação de pagamentos, mas foca no setor agrícola e não explora desafios e oportunidades específicos da venda de apólices.

Liu et al. (2019) discute o uso do blockchain em seguros de viagem para garantir integridade de dados, mas limita-se à proteção de informações transacionais, sem abordar o potencial dos contratos inteligentes na venda de apólices.


\subsection{Implementação de Blockchain e Contratos Inteligentes}\label{sec2.3}

Shetty et al. (2022) observam que algoritmos criptográficos e processos digitais sofisticados são essenciais para proteger informações e garantir a autenticidade das transações. Os contratos inteligentes, segundo esses autores, possuem quatro características principais: formulário digital, cláusulas embutidas no código, desempenho controlado tecnologicamente e execução ininterrupta, salvo condições não atendidas. Embora reconheçam os benefícios desses contratos para segurança e automação, apontam-se desafios na aplicação prática ao setor de seguros, como a necessidade de flexibilidade para lidar com exceções, a importância da intervenção humana em disputas e exigências regulatórias, e a adaptação cultural e tecnológica de seguradoras e clientes.

Aleksieva et al. (2020a) apresentam o Hyperledger Fabric como uma plataforma blockchain adequada para consórcios empresariais, destacando sua flexibilidade e capacidade de adaptação. Contudo, o texto não aprofunda os desafios operacionais e de interoperabilidade na venda de apólices, como a coordenação entre múltiplos participantes e a integração com sistemas legados. Aleksieva et al. (2020b) reforçam a flexibilidade do Hyperledger Fabric, mas deixam de abordar como a tecnologia pode melhorar o relacionamento com o cliente final e personalizar apólices, além de não discutir questões de privacidade e custos operacionais ao comparar Hyperledger e Ethereum.

Xiao et al. (2020) propõem a integração de IoT e blockchain para seguros de veículos, permitindo preços dinâmicos e customizados. No entanto, não detalham como essa integração pode otimizar a venda de apólices, nem abordam desafios como interoperabilidade entre blockchains públicos e privados e custos de transação.

Bai et al. (2022) destacam o impacto positivo do blockchain na automação de processos internos, como o pagamento de sinistros e prevenção de fraudes, mas não aprofundam a aplicação na venda de apólices, nem discutem desafios como integração com sistemas legados e conformidade regulatória.

Liu et al. (2019) ressaltam a eficiência dos contratos inteligentes, mas alertam para riscos de erros irreversíveis. O texto sugere que seria importante discutir mecanismos de revisão e validação, além de explorar a automação de todo o ciclo de venda de apólices.

Por fim, Klapkiv e Kedra (2022) observam que os benefícios das inovações tecnológicas variam conforme o momento de implementação e o porte da empresa, mas não abordam desafios específicos do setor de seguros, como integração tecnológica e proteção de dados sensíveis.


\subsection{Escalabilidade no Processamento de Transações da Blockchain}\label{sec2.4}
Segundo Shetty et al. (2022), a implementação de contratos inteligentes oferece oportunidades para inovação e melhoria de desempenho nas empresas, como maior envolvimento do cliente, automação de pagamentos e prevenção de disputas. ALEKSIEVA et al. (2020a) complementa que a blockchain otimiza processos de resseguro e operações de corretoras, além de melhorar o monitoramento em áreas como medicamentos e seguros de saúde.

No entanto, há limitações importantes quanto à escalabilidade das redes públicas de blockchain, como Bitcoin e Ethereum, que apresentam baixo desempenho em termos de transações por segundo (TPS), com Ethereum atingindo apenas cerca de 15 a 20 TPS. Isso é insuficiente para grandes volumes de vendas de apólices. Blockchains privadas ou permissionadas, como Hyperledger Fabric, são mais adequadas para cenários de alto TPS, mas o texto não aprofunda como garantir desempenho sem comprometer segurança e transparência. A integração entre blockchain e IoT é vista como promissora para escalabilidade e privacidade, mas faltam análises específicas sobre sua aplicação na venda de apólices, especialmente quanto ao processamento rápido de transações e proteção de dados sensíveis dos segurados.

Xiao et al. (2020) propõe um modelo híbrido entre Hyperledger Fabric e Ethereum para equilibrar desempenho e funcionalidades, sugerindo o agendamento de processos complexos em horários de menor demanda. Contudo, essa abordagem pode não ser ideal para vendas de apólices, que exigem respostas em tempo real, principalmente em vendas online ou campanhas de marketing. O uso de Oracles para integrar dados externos é prático, mas traz preocupações quanto à confiabilidade e segurança, que não são abordadas no texto. Tecnologias como sharding e rollups poderiam ser discutidas como alternativas para garantir processamento eficiente sem comprometer a experiência do cliente.

ALEKSIEVA et al. (2020b) destaca as vantagens das blockchains privadas, como custos operacionais mais baixos e validação rápida de transações, mas não explora os desafios de escalabilidade em grandes volumes de vendas. A falta de interoperabilidade pode criar silos de dados e dificultar a automação completa de processos. O uso de blockchains públicas para pagamentos automáticos vinculados a criptomoedas também não considera a volatilidade dessas moedas, um obstáculo para adoção em larga escala.

Bai et al. (2022) enfatiza o potencial da IoT para coleta de dados em tempo real, personalização de produtos e mitigação de riscos, especialmente no setor agrícola. A colaboração entre IoT e blockchain pode resolver problemas de segurança, escalabilidade e confiabilidade, mas o foco está no processamento de sinistros, não na venda de apólices. Soluções como sharding, sidechains ou redes híbridas poderiam ser exploradas para garantir escalabilidade na venda de apólices.

Liu et al. (2019) ressalta a segurança e rastreabilidade do blockchain, essenciais para combater fraudes no setor de seguros. Entretanto, não aborda como lidar com grandes volumes de transações na venda de apólices, nem como contratos inteligentes podem otimizar esse processo em tempo real.

ALEKSIEVA et al. (2020c) traça a evolução do blockchain desde o Bitcoin, destacando aplicações em diversos setores, incluindo seguros, com foco na otimização de sinistros e redução de custos. Falta, porém, análise sobre escalabilidade e automação na venda de apólices, ponto central da dissertação.

Por fim, Klapkiv e Kedra (2022) discutem a redução da assimetria de informações e custos de transação via tecnologias, mas concentram-se nas economias pós-venda, sem abordar como contratos inteligentes podem transformar a emissão e assinatura de apólices. Soluções como blockchains híbridas ou sidechains poderiam ser exploradas para processar grandes volumes de transações de forma eficiente.



\section{Definição da Arquitetura Descentralizada}\label{sec3}

Nesta seção do artigo, delimita-se a aplicação da tecnologia blockchain no domínio de seguros, especificamente para o processamento de vendas de apólices, com suporte ao registro de transações por meio de contratos inteligentes.

O diagrama de arquitetura apresentado, na Figura \ref{fig:fig3}, adota uma abordagem descentralizada, estruturada de forma técnica para suportar uma aplicação baseada na blockchain Hyperledger Fabric voltada à comercialização de apólices de seguro. Essa perspectiva técnica é essencial para demonstrar com clareza os componentes, camadas e fluxos envolvidos na validação e registro de transações via contratos inteligentes, promovendo a transparência, segurança e rastreabilidade inerentes ao contexto do seguro.

A arquitetura detalha as interações entre os nós da rede, a infraestrutura Hyperledger Fabric, os contratos inteligentes e os mecanismos de validação, evidenciando como esses elementos se integram para garantir a confiabilidade e integridade do sistema através dos seguintes componentes:

\begin{itemize}
    \item insurance-sales-api: API de vendas de seguros representado em azul, este componente é a interface da aplicação que gerencia as transações relacionadas à venda de apólices de seguro. Ele atua como o ponto de entrada para os usuários e sistemas externos, facilitando a comunicação com a rede blockchain;
\end{itemize}

\begin{itemize}
    \item Gateway: composto por dois contratos inteligentes, um para o peer0-org1 e outro para peer0-org2, este módulo serve como um intermediário entre a API de vendas de seguros e a infraestrutura Hyperledger Fabric. Os contratos inteligentes contêm a lógica de negócios necessária para validar e registrar transações na blockchain;
\end{itemize}

\begin{itemize}
    \item Org1 e Org2: a rede Hyperledger Fabric é composta por duas organizações, cada uma com seus próprios peers e funções específicas:
    \begin{itemize}
        \item Organização 1 (Org1 - Seguradora): representa a seguradora, a entidade responsável por definir e gerenciar as regras de elegibilidade para a aquisição das apólices de seguro;
    \end{itemize}     
    \begin{itemize}
        \item O peer peer0-org1 armazena o ledger da organização e executa os contratos inteligentes relacionados às políticas da seguradora;
    \end{itemize}
    \begin{itemize}
        \item Conecta-se ao Gateway, garantindo a execução das regras de negócio antes do registro da transação na blockchain;
    \end{itemize}  
    \begin{itemize}
        \item Organização 2 (Org2 - Corretor de Seguros): representa um corretor de seguros e colabora na intermediação e venda de apólices;
    \end{itemize}   
    \begin{itemize}
        \item O peer peer0-org2 mantém uma cópia do ledger e executa contratos inteligentes específicos para validar e intermediar as transações;
    \end{itemize}
    \begin{itemize}
        \item Trabalha em conjunto com a seguradora para garantir a aplicação correta das regras;
    \end{itemize}

\end{itemize}

\begin{itemize}
    \item Infraestrutura Fabric: representa a estrutura principal da rede blockchain Hyperledger Fabric, composta pelos seguintes elementos:
    \begin{itemize}
        \item peer0-org1 e peer0-org2 (Peers das Organizações 1 e 2): responsáveis por validar e armazenar as transações na blockchain. Cada peer possui cópias do livro-razão (ledger) e pode executar contratos inteligentes;
    \end{itemize}   
    \begin{itemize}
        \item Orderer: componente responsável pela ordenação e propagação das transações dentro da rede blockchain, garantindo a consistência e a confiabilidade dos registros;
    \end{itemize}
    \begin{itemize}
        \item CA (Certificate Authority): autoridade certificadora encarregada de emitir credenciais de identidade para os participantes da rede, garantindo autenticação e segurança;
    \end{itemize}
    \begin{itemize}
        \item CouchDB: banco de dados utilizado como repositório de estados para os peers, permitindo consultas avançadas e facilitando a interação com os dados armazenados no ledger;
    \end{itemize}    
\end{itemize}

\begin{itemize}
    \item Hyperledger Explorer: ferramenta de monitoramento e visualização dos dados e transações registradas na rede Hyperledger Fabric. Inclui:
    \begin{itemize}
        \item Hyperledger Explorer Console: interface gráfica que permite aos administradores e usuários visualizar transações, blocos e a estrutura da rede;
    \end{itemize}
    \begin{itemize}
        \item Explorer DB: banco de dados utilizado pelo Hyperledger Explorer para armazenar e consultar informações sobre as transações e estados da blockchain.
    \end{itemize}
\end{itemize}

\begin{figure}[h]
    \centering{\includegraphics[width=1.00\linewidth]{images/proposalarchitecture.png}}
    \caption{Diagrama de arquitetura da aplicação proposta}
    \label{fig:fig3}
\end{figure}

A rede blockchain foi criada por meio do Docker, dado que ele oferece flexibilidade para customizar a infraestrutura de acordo com a necessidade de cada container, como:
\begin{enumerate}[label=\alph*)]
    \item Localização dos nós;
    \item CPU e RAM do hardware;
    \item Endorsers necessários para o consenso;
    \item Adição de novas organizações e membros à rede.
\end{enumerate}

\subsection{Etapas do Fluxo de Dados}

O diagrama de fluxo de dados, apresentado na Figura \ref{fig:fig4}, ilustra o processo de compra de uma apólice de seguro, desde a solicitação inicial pelo usuário até a confirmação da apólice. A finalidade é fornecer uma visão clara e concisa de como a transação é gerenciada e monitorada dentro do sistema conforme as seguintes etapas:
    
\begin{enumerate}[label=\alph*)]
    \item Usuário solicita a compra de uma apólice: a aplicação cliente acessa a aplicação insurance-sales-api para solicitar um seguro;
    
    \item A API de vendas envia a solicitação ao Gateway: a API envia os dados do seguro ao Gateway, que decide para qual organização direcionar a transação;

    \item Execução dos contratos inteligentes nos peers: o Gateway chama os contratos inteligentes nos peers (peer0-org1 e peer0-org2) para validar regras de negócios;
    
    \item Processamento da transação na blockchain: se os critérios forem atendidos, a transação é enviada ao Orderer, que a adiciona ao ledger, caso contrário, a solicitação é rejeitada;
    
    \item Armazenamento dos dados: os peers registram a apólice no ledger e no CouchDB, permitindo consultas futuras;
    
    \item Consulta e monitoramento via Hyperledger Explorer: a seguradora e os administradores podem visualizar as transações por meio do Hyperledger Explorer;
    
    \item Confirmação da apólice ao usuário: após a transação ser registrada, a API notifica o usuário sobre a conclusão do processo.
\end{enumerate}

\begin{figure}[h]
    \centering
    \includegraphics[width=1\linewidth]{images/dataflow.jpg}
    \caption{Diagrama de Fluxo de Dados}
    \label{fig:fig4}
\end{figure}

O fluxo apresentado na Figura \ref{fig:fig4} detalha as interações entre os diferentes componentes do sistema, incluindo a API de vendas, o Gateway, os peers da blockchain, o Orderer, o CouchDB e o Hyperledger Explorer. Cada etapa do processo é representada, mostrando como os dados são transmitidos e processados para garantir a validação e o registro da apólice de seguro.

\subsection{Adoção do Hyperledger Fabric}

A adoção do Hyperledger Fabric neste artigo baseou-se no trabalho de Androulaki et al. (2018) que indica que o Hyperledger Fabric é um sistema operacional distribuído para blockchains privados que executa aplicativos distribuídos escritos em linguagens de programação de uso geral (por exemplo, Go, Java, Node.js). Ele rastreia com segurança seu histórico de execução em uma estrutura de dados e não possui criptomoeda integrada.

Diferentes aplicações, como as voltadas para IoT, saúde e seguros, podem ser integradas ao chaincode. Conforme exemplificado na Figura 25, o chaincode pode ser executado como um contrato inteligente, ao mesmo tempo que oferece conectividade com diferentes APIs. A execução de um contrato inteligente requer algumas etapas primárias na rede blockchain. Em primeiro lugar, o chaincode é executado a partir do Chaincode Developer Kit (CDK), depois os contratos inteligentes são entregues aos nós endorsers que realmente endossam a validade do contrato e permitem ainda a execução do contrato. Depois de confirmar a legitimidade do titular do contrato. Depois disso, o chaincode transfere a transação para os nós orderes que a combinam e geram os blocos de acordo com o tamanho de bloco predefinido. São computados os hashes dos blocos que depois são adicionados à cadeia por meio do mecanismo de consenso. O status do livro razão é mantido consistentemente desta forma usando Byzantine Fault Tolerance (BFT), Kafka, Solo etc. (ALI SYED et al., 2019).

\begin{figure}[h]
    \centering
    \includegraphics[width=1\linewidth]{images/HPFModel.jpg}
    \caption{Modelo Hyperledger Fabric}
    \label{fig:fig5}
\end{figure}

Com relação aos mecanismos de segurança, o Hyperledger Fabric, apresenta a possibilidade de coletas privadas de dados, e permitem que determinados participantes autorizados acessem apenas dados específicos.

O Fabric está se tornando uma arquitetura de livro razão favorável para aplicações gerais. Ele fornece uma interface de contrato inteligente para o desenvolvimento de aplicativos chamada chaincode. Chaincode pode ser desenvolvido em várias linguagens, como NodeJS, Go, Java, TypeScript. Ele também fornece uma interface Restful para aplicativos existentes se conectarem à rede blockchain, possibilitando o uso de dados off-chain.


\section{Implementação}
O sistema proposto utiliza a tecnologia blockchain Hyperledger Fabric para registrar transações de vendas de apólices de seguro viagem, empregando contratos inteligentes. A solução envolve participantes como a seguradora e a apólice, e sua arquitetura foi desenvolvida para ser flexível e adaptável, permitindo ajustes conforme as necessidades e especificidades de cada aplicação no segmento de seguros.

\subsection{Configuração da Rede Hyperledger Fabric}
A rede Hyperledger Fabric envolve a infraestrutura necessária para suportar a comunicação e a validação das transações entre os diferentes participantes da rede. Isso inclui a geração de certificados e chaves criptográficas, a definição de políticas de governança e a configuração dos canais de comunicação. Etapas para configuração da rede Hyperledger Fabric:
\renewcommand{\labelitemi}{\tiny$\bullet$}
\begin{itemize}
    \item Geração de artefatos de cripto material: os certificados e chaves criptográficas necessárias para identificar e autenticar todos os participantes da rede, garante a segurança e a integridade das transações com o uso da ferramenta cryptogen conforme indicado por Hyperledger (2020), e habilita a rede blockchain  para testes;
    \item Definição do arquivo de configuração: permite estruturar a rede, incluindo os canais, as organizações participantes, as políticas de consenso, e as configurações do orderer e do peer. Este arquivo de configuração também define como os componentes da rede se comunicam e interagem entre si;
    \item Criação de contêineres Docker: os peers, orderers e CA com o uso do Docker Compose são os componentes necessários da rede;
    \item Iniciar os contêineres: considera os serviços configurados no Docker Compose para que a rede Hyperledger Fabric esteja pronta para operação.
\end{itemize}
\renewcommand{\labelitemi}{\textbullet}


\subsection{Implementação do Contrato Inteligente}
O desenvolvimento do contrato inteligente considera uma operação que envolve uma apólice de seguro-viagem atendendo a estrutura para o contrato de seguro que possa ser armazenado e manipulado na rede blockchain.

A classe Eligibility é responsável por validar as operações de criação, atualização e cancelamento de apólices de seguro. Ela contém os métodos, indicados a seguir, que verificam se os dados fornecidos são válidos e se as regras de negócio são respeitadas.

\renewcommand{\labelitemi}{\tiny$\bullet$}
\begin{itemize}
    \item Validate policy creation: valida a criação de uma apólice verificando datas, destino e número de passageiros;
    \item Validate  policy update: valida a atualização de uma apólice verificando o novo status;
    \item Validate policy cancellation: valida o cancelamento de uma apólice verificando se está ativa para que possa ser cancelada.
\end{itemize}
\renewcommand{\labelitemi}{\textbullet}

A classe Policy é um contrato inteligente que gerencia apólices de seguro. Ela contém métodos para criar, consultar, listar, atualizar e cancelar apólices, além de calcular reembolsos.

% \begin{figure}
%     \centering
%     \includegraphics[width=1\linewidth]{images/SmartContract.png}
%     \caption{Enter Caption}
%     \label{fig:placeholder}
% \end{figure}

\renewcommand{\labelitemi}{\tiny$\bullet$}
\begin{itemize}
    \item Init: inicializa o chaincode;
    \item Create policy: cria apólice a partir de um JSON;
    \item Query policy: consulta uma apólice pelo ID;
    \item Query all policies: lista todos os IDs de apólices;
    \item Query by holder document id: consulta apólices pelo documento do titular;
    \item Update policy status: atualiza o status de uma apólice;
    \item Cancel policy: cancela uma apólice e processa o reembolso;
    \item Calculate refund: calcula o valor do reembolso com base na data de cancelamento e no prêmio.
\end{itemize}
\renewcommand{\labelitemi}{\textbullet}

As transações específicas, como a compra do seguro e processamento das regras envolvidas, como a criação de uma apólice verificando datas, destino e número de passageiros para validação e registro corretamente na blockchain foi desenvolvido com a linguagem de programação JAVA.

As classes usadas para representar e manipular dados relacionados a apólices de seguro dentro do contrato inteligente na rede Hyperledger Fabric são descritas a seguir:

\renewcommand{\labelitemi}{\tiny$\bullet$}
\begin{itemize}
    \item InsuranceProduct.java: a classe InsuranceProduct representa um produto de seguro, contendo informações sobre a cobertura e o valor segurado;
    \item TravelInfo.java: a classe TravelInfo representa informações de viagem associadas a uma apólice de seguro e armazena detalhes da viagem, como destino, datas de início e término, e número de passageiros;
    \item Refund.java: a classe Refund representa um reembolso associado ao cancelamento de uma apólice de seguro, representa os detalhes de um reembolso, incluindo o ID da apólice, o titular da apólice, o valor do reembolso e o status da apólice;
    \item Policy.java: a classe Policy representa uma apólice de seguro, contendo informações detalhadas sobre a apólice, o titular, o produto de seguro, os passageiros e as informações de viagem;
    \item Passenger.java: a classe Passenger representa um passageiro associado a uma apólice de seguro, incluindo detalhes pessoais como nome, documento, data de nascimento, e informações de contato.
    \item Essas classes são usadas em conjunto para representar e manipular dados relacionados a apólices de seguro dentro de um contrato inteligente na rede Hyperledger Fabric. Elas permitem a criação, validação, atualização e cancelamento de apólices, bem como a serialização e desserialização de dados para armazenamento e recuperação na blockchain.
\end{itemize}
\renewcommand{\labelitemi}{\textbullet}

Essas classes são usadas em conjunto para representar e manipular dados relacionados a apólices de seguro dentro de um contrato inteligente na rede Hyperledger Fabric. Elas permitem a criação, validação, atualização e cancelamento de apólices, bem como a serialização e desserialização de dados para armazenamento e recuperação na blockchain.

\subsection{Interface para Integração}

A integração com os contratos inteligentes na blockchain para registrar informações como a apólice de seguro, contendo a lógica de negócios para criar, consultar e tomar decisões para registro das transações baseadas no estado do contrato, foi feita a partir de uma aplicação Spring Boot que fornece uma API REST para gerenciar seguros de viagem.

O desenvolvimento da API REST considera, como apresentado a seguir, as principais classes com suas respectivas responsabilidades.
\renewcommand{\labelitemi}{\tiny$\bullet$}
\begin{itemize}
    \item PolicyController.java: controlador REST que expõe endpoints para gerenciar apólices de seguro;
    \item PolicyService.java: serviço que contém a lógica de negócios para gerenciar apólices de seguro;
    \item WalletService.java: serviço que gerencia a carteira de identidades do Hyperledger Fabric.
\end{itemize}
\renewcommand{\labelitemi}{\textbullet}

A classe PolicyController é um controlador REST que expõe endpoints para gerenciar apólices de seguro. Ela utiliza o serviço PolicyService, de acordo com os endpoints a seguir, para realizar operações de criação, consulta, atualização e cancelamento de apólices.

\renewcommand{\labelitemi}{\tiny$\bullet$}
\begin{itemize}
    \item POST /api/policies/create: cria apólice de seguro;
    \item GET /api/policies/{policyId}: consulta uma apólice de seguro pelo ID;
    \item GET /api/policies/findAll: consulta todas as apólices de seguro;
    \item GET /api/policies/holder/{documentId}: consulta apólices de seguro pelo documento do titular;
    \item PUT /api/policies/{policyId}/status: atualiza o status de uma apólice de seguro;
    \item POST /api/policies/cancel/{policyId}: cancela uma apólice de seguro e processa o reembolso.
\end{itemize}
\renewcommand{\labelitemi}{\textbullet}

A classe PolicyService contém a lógica de negócios para gerenciar apólices de seguro, de acordo com os métodos apresentados a seguir. É responsável por criar, consultar, atualizar e cancelar apólices, além de processar reembolsos.

\renewcommand{\labelitemi}{\tiny$\bullet$}
\begin{itemize}
    \item Create policy: cria apólice de seguro;
    \item Query policy: consulta uma apólice de seguro pelo ID;
    \item Query all policies: consulta todas as apólices de seguro;
    \item Query by holder documentId: consulta apólices de seguro pelo documento do titular;
    \item Update policy status: atualiza o status de uma apólice de seguro;
    \item Cancel policy: cancela uma apólice de seguro e processa o reembolso.
\end{itemize}
\renewcommand{\labelitemi}{\textbullet}

A classe WalletService é responsável por gerenciar a carteira de identidades do Hyperledger Fabric. Ela lida com operações relacionadas à autenticação e autorização na rede Hyperledger Fabric, incluindo a gestão de credenciais e a interação com a rede blockchain. A classe WalletService inclui métodos como:

\renewcommand{\labelitemi}{\tiny$\bullet$}
\begin{itemize}
    \item Load credentials: carrega as credenciais de um usuário específico;
    \item Store credentials: armazena as credenciais de um usuário específico;
    \item Authenticate: autentica um usuário na rede Hyperledger Fabric;
    \item Authorize: autoriza um usuário a acessar um recurso específico na rede Hyperledger Fabric.
\end{itemize}
\renewcommand{\labelitemi}{\textbullet}


\subsection{Frontend para interação de usuários}

O desenvolvimento da aplicação web teve como finalidade viabilizar a interação de usuários com uma rede blockchain voltada à gestão de apólices de seguro-viagem. Desenvolvida com a biblioteca React, a solução proporcionou funcionalidades como a criação, a consulta de status e o cancelamento de apólices de forma segura e transparente. A comunicação entre o frontend e a rede blockchain ocorreu por meio de uma API backend, a qual foi responsável por intermediar as requisições e garantir a correta execução das operações junto aos contratos inteligentes. A Figura \ref{fig:fig6}, apresentada a seguir, exibe a tela para executar o processo de criação de apólice, assim como os menus que permite executar os processos de busca e cancelamento de apólice.

\begin{figure}[h]
    \centering
    \includegraphics[width=1\linewidth]{images/frontend.jpg}
    \caption{Frontend para interação de usuários}
    \label{fig:fig6}
\end{figure}

Os procedimentos para construir e executar o frontend, apresentado na Figura \ref{fig:fig6}, buscaram assegurar a integridade, a rastreabilidade e a confiabilidade dos dados manipulados no sistema com o uso das melhores praticas.

\subsection{Ambiente de execução}
%2.3.6	Resultado da Implementação da Prova de Conceito

A Figura \ref{fig:fig7}, mostra a lista de containers em execução, cada um dos containers, indicados a seguir, representa um componente diferente do sistema de vendas de apólices de seguro.

\renewcommand{\labelitemi}{\tiny$\bullet$}
\begin{itemize}
    \item dev-insurance-sales-api: este container executa a API de vendas de seguros, que é responsável por receber as solicitações de compra de apólices dos usuários;
    \item dev-peer0.org2.example.com-insurance-chaincode: este container representa um peer da organização 2, que executa o contrato inteligente relacionado ao seguro. Ele participa da validação das transações;
    \item dev-peer0.org1.example.com-insurance-chaincode: similar ao anterior, este container representa um peer da organização 1, que também executa o contrato inteligente relacionado ao seguro e participa da validação das transações;
    \item cli: este container executa ferramentas de linha de comando do Hyperledger Fabric, usadas para interagir com a rede blockchain;
    \item peer0.org2.example.com: este container representa um peer da organização 2, que mantém uma cópia do ledger e participa do processo de consenso;
    \item peer0.org1.example.com: este container representa um peer da organização 1, que também mantém uma cópia do ledger e participa do processo de consenso;
    \item orderer.example.com: este container executa o serviço de ordenação do Hyperledger Fabric, que é responsável por ordenar as transações e adicioná-las ao ledger;
    \item couchdb0 e couchdb1: estes containers executam instâncias do CouchDB, que são usadas como bancos de dados para armazenar o estado global dos peers, permitindo consultas eficientes.
    \item ca\_org2, ca\_org1 e ca\_orderer: estes containers executam as Autoridades de Certificação (CAs) para as organizações 1 e 2, e para o serviço de ordenação, respectivamente. Eles são responsáveis por emitir certificados de identidade para os componentes da rede.
\end{itemize}
\renewcommand{\labelitemi}{\textbullet}

\begin{figure}[h]
    \centering
    \includegraphics[width=1\linewidth]{images/containers.png}
    \caption{Containers implementados para ambiente de execução}
    \label{fig:fig7}
\end{figure}

Cada um desses containers desempenha uma função essencial na operação e na manutenção da rede blockchain, assegurando que as transações relativas às apólices de seguro sejam tratadas de forma segura, confiável e eficiente, em conformidade com os requisitos de integridade, rastreabilidade e desempenho do sistema distribuído.

\section{Resultados}
Nesta seção, são apresentados e discutidos os resultados obtidos nos testes conduzidos para validar a segurança nas transações de venda de apólices de seguros e a eficiência do sistema, com ênfase em sua estabilidade diante do aumento do volume de transações. Os resultados foram analisados de modo a verificar a eficácia dos mecanismos de controle de fraude por meio da detecção de duplicação de transações, confirmando que o contrato inteligente implementa funcionalidades capazes de identificar e rejeitar transações redundantes. Adicionalmente, foram examinados fatores relacionados ao desempenho do sistema, como a taxa de processamento medida em número de transações por segundo.

Cada cenário de teste é descrito de forma a explicitar os desafios enfrentados e as soluções adotadas durante a execução da transação de criação de apólices. A análise busca identificar de que maneira variáveis, tais como o volume de transações e os mecanismos de controle, influenciam o desempenho e a segurança do sistema. Os resultados são discutidos com foco nas transações de venda de apólices de seguros e na avaliação da viabilidade técnica da arquitetura proposta.

\subsection{Análise de Segurança}
A integridade da apólice de seguro constitui requisito fundamental para assegurar a confiabilidade, a segurança e a imutabilidade das informações registradas na blockchain. Com o objetivo de avaliar a capacidade do sistema em proteger a apólice contra acessos não autorizados e contra adulterações, foram elaborados três cenários de teste que analisam distintos aspectos relacionados à segurança e à proteção dos dados na rede Hyperledger Fabric.

Os cenários apresentados na Table \ref{tab:tab1} têm como finalidade verificar se as políticas de autenticação e autorização estão corretamente implementadas, bem como se o sistema mantém a consistência dos dados armazenados na blockchain, impedindo modificações indevidas. Busca-se, assim, avaliar a capacidade do sistema em salvaguardar a apólice de seguro contra acessos não autorizados e adulterações, garantindo que esta permaneça íntegra ao longo de todas as etapas do processo.

Os cenários de teste permitiram validar se a aplicação preserva a segurança e a integridade das apólices de seguro, prevenindo acessos indevidos e alterações não autorizadas. Por meio dessa abordagem, foi possível evidenciar que os contratos armazenados na blockchain mantêm-se confiáveis, imutáveis e acessíveis exclusivamente aos participantes devidamente autorizados pelo sistema.

Os testes comprovam a eficácia da aplicação na proteção e controle de apólices de seguro: no Cenário 1, as 50 tentativas de modificação não autorizada foram bloqueadas; no Cenário 2, todas as transações legítimas foram processadas corretamente; e no Cenário 3, 50 tentativas de acesso indevido foram impedidas e registradas como eventos suspeitos, confirmando a integridade e a segurança do sistema.

\begin{sidewaystable}%
\centering
\caption{Cenários de teste para segurança, controle de acesso e integridade dos dados da apólice de seguro viagem.\label{tab:tab1}}%
\small
\begin{tabular*}{\textheight}{@{\extracolsep\fill}
p{3.1cm}   % Cenário
p{3.7cm}   % Objetivo
p{4.4cm}   % Procedimento
p{4.4cm}   % Critérios de Sucesso
p{4.5cm}   % Resultados
@{\extracolsep\fill}}%
\toprule
\textbf{Cenário} &
\textbf{Objetivo} &
\textbf{Procedimento} &
\textbf{Critérios de Sucesso} &
\textbf{Resultados} \\
\midrule

\textbf{Cenário 1:} Tentativa de modificação não autorizada da apólice
&
Garantir que apenas usuários autorizados possam modificar os dados da apólice.
&
Criar uma apólice com usuário autorizado; tentar alterar seus dados com usuário não autorizado; monitorar respostas da API e logs da blockchain.
&
O sistema deve bloquear a modificação, registrar a tentativa e manter a versão original da apólice na blockchain.
&
A apólice permaneceu inalterada após tentativas não autorizadas, demonstrando que a arquitetura restringe corretamente modificações apenas a usuários autorizados.
\\

\textbf{Cenário 2:} Verificação da integridade dos dados da apólice
&
Assegurar que os dados da apólice não sejam alterados fora do fluxo transacional autorizado.
&
Criar a apólice e registrar seu hash na blockchain; realizar operações normais; comparar o hash original com o hash atual.
&
O hash da apólice deve permanecer inalterado, exceto em modificações legítimas.
&
Todas as transações executadas preservaram a integridade dos dados, mantendo consistência entre os hashes armazenados e os registros da apólice.
\\

\textbf{Cenário 3:} Acesso indevido à apólice de seguro
&
Testar a robustez do sistema contra tentativas de acesso não autorizado a dados sensíveis.
&
Criar a apólice com permissões restritas; simular acessos não autorizados via API e com credenciais inválidas; analisar respostas e logs.
&
O sistema deve bloquear o acesso não autorizado e registrar eventos suspeitos.
&
As tentativas de acesso indevido foram bloqueadas com sucesso e devidamente registradas, evidenciando a robustez do controle de acesso.
\\

\bottomrule
\end{tabular*}
\end{sidewaystable}

\subsection{Análise de Desempenho}

Esta subseção apresenta uma avaliação do desempenho do processo de comercialização de apólices de seguro, em termos de tempo de resposta, para a arquitetura proposta. O objetivo é identificar de que forma o número de transações por segundo e os volumes de dados influenciam o comportamento do sistema em condições operacionais reais.

A análise de desempenho tem como finalidade mensurar a eficiência da aplicação voltada à venda de apólices de seguro, considerando tanto o tempo de resposta quanto a escalabilidade do sistema sob diferentes perfis de carga de trabalho. Os testes descritos na Tabele \ref{tab:tab2} foram concebidos para quantificar a capacidade do sistema de processar transações em alta taxa e sob grandes volumes de dados, assegurando um funcionamento adequado em cenários de uso realista.

Os três cenários de teste permitiram avaliar de maneira sistemática a eficiência e a escalabilidade do sistema em distintas condições operacionais, verificando se a aplicação descentralizada é capaz de suportar altas cargas de transações, grandes volumes de dados e executar consultas de forma eficiente.

Os testes de desempenho foram realizados em ambiente controlado para cada um dos 3 cenários, e os resultados validaram a eficiência da aplicação na venda de apólice de seguros para a aplicação proposta.

Para o Cenário 1, em que o teste para medir o tempo médio de resposta para o processamento de uma única transação de venda de apólice e garantir que o tempo esteja dentro dos limites aceitáveis, foi realizado a criação de 1000 apólices individuais utilizando um payload com 332 bytes para cada requisição. O tempo médio total para criação de uma apólice foi de 2140 ms. O tempo máximo apresentado para uma transação foi de 4939 ms, não atendendo ao critério de sucesso previamente determinado, possivelmente devido a variações na conexão com a blockchain para a primeira requisição realizada. A mediana do tempo total ficou muito próximo da média, em 2124 ms, indicando uma distribuição equilibrada dos tempos de processamento. O tempo mínimo registrado foi de 2078 ms. Para as 1000 requisições realizadas foi alcançado uma taxa de sucesso de 99,9\% mostrando estabilidade na execução das transações.

Para o Cenário 2, com a finalidade de avaliar o desempenho do sistema sob carga crescente, medindo o impacto no tempo de resposta à medida que o número de transações simultâneas aumenta, diferentes níveis de carga foi simulado, iniciando com 100 transações por segundo e aumentando progressivamente para 500, 1000 e 2000 transações por segundo. A Figura \ref{fig:fig8}, apresenta o resultado para os testes de escalabilidade.


\begin{sidewaystable}%
\centering
\caption{Cenários de Teste para Avaliação do Desempenho da Aplicação.\label{tab:tab2}}%
\small
\begin{tabular*}{\textheight}{@{\extracolsep\fill}
p{3.1cm}   % Cenário
p{4.7cm}   % Objetivo
p{6.9cm}   % Procedimento
p{6.9cm}   % Critérios de Sucesso
@{\extracolsep\fill}}%
\toprule
\textbf{Cenário} &
\textbf{Objetivo} &
\textbf{Procedimento} &
\textbf{Critérios de Sucesso} \\
\midrule

\textbf{Cenário 1:} Teste de Tempo de Resposta por Transação
&
Medir o tempo médio de resposta para o processamento de uma única transação de venda de apólice e garantir que o tempo esteja dentro dos limites aceitáveis.
&
Simular a criação de 1000 apólices individuais, uma por vez, registrando o tempo necessário para cada operação.
Monitorar o tempo médio de resposta do contrato inteligente na blockchain.
Avaliar se o tempo de resposta se mantém estável à medida que o volume de transações aumenta.
&
O tempo médio de resposta por transação deve estar dentro dos limites aceitáveis (menor a 3 segundos, de acordo com experiência empírica) para garantir a fluidez da aplicação, sem atrasos significativos.
\\

\textbf{Cenário 2:} Teste de Escalabilidade com Carga Aumentada
&
Avaliar o desempenho do sistema sob carga crescente, medindo o impacto no tempo de resposta à medida que o número de transações simultâneas aumenta.
&
Simular diferentes níveis de carga, iniciando com 100 transações por segundo e aumentando progressivamente para 500, 1000 e 2000 transações por segundo.
Monitorar a latência, uso de CPU, memória e tempo de commit das transações na blockchain.
Identificar o ponto de saturação do sistema e se há degradação significativa no desempenho.
&
O sistema deve manter o número de TPS sem degradação. Caso haja degradação, deve ser identificada a capacidade máxima suportada antes de comprometer a usabilidade.
\\

\textbf{Cenário 3:} Teste de Desempenho com Grandes Volumes de Dados
&
Avaliar o impacto da manipulação de grandes volumes de dados no tempo de resposta e na eficiência do sistema.
&
Criar um banco de dados simulado contendo mais de 100 mil apólices.
Executar consultas complexas, como busca por titular e listagem de apólices.
Medir o tempo de resposta para cada consulta e verificar a eficiência da busca em grandes volumes de dados.
&
Baseado em experiencia empírica, o sistema deve responder em menos de 3 segundos às consultas, garantindo a escalabilidade e a eficiência da arquitetura para processar grandes quantidades de informações sem comprometer a performance.
\\

\bottomrule
\end{tabular*}
\end{sidewaystable}

A primeira simulação, com 100 apólices, apresentou tempo médio total de 2280 ms para a criação de uma apólice, valor muito próximo ao observado no Cenário 1, de 2140 ms. A mediana foi de 2233 ms, próxima da média, o que indica consistência no processo de venda de apólices. O tempo mínimo observado foi de 973 ms e o máximo de 3636 ms, resultando em uma amplitude de 2663 ms, atribuída principalmente ao tempo de conexão e resposta da blockchain, evidenciando maior oscilação à medida que se eleva o volume de transações.

A estabilidade das transações torna-se mais perceptível com o aumento do volume para 500, 1000 e 2000 transações por segundo. Observa-se, entretanto, uma taxa de sucesso de 99,8\% nas simulações com 500 e 1000 transações por segundo e de 99,6\% na simulação com 2000 transações por segundo, sendo esta redução associada a falhas na conexão com a blockchain. 

O maior throughput foi obtido na simulação com 500 apólices, com 27,8 transações processadas por segundo. Contudo, a simulação com 100 apólices apresentou throughput de 25,0 transações por segundo com taxa de sucesso de 100\%, o que demonstra maior estabilidade operacional no processo de venda de apólices.

\begin{figure}[h]
    \centering
    \includegraphics[width=1\linewidth]{images/escalabilidade.jpg}
    \caption{Resultado do Teste de Escalabilidade}
    \label{fig:fig8}
\end{figure}

Para o Cenário 3, com o objetivo de avaliar o impacto da manipulação de grandes volumes de dados sobre o tempo de resposta e a eficiência do sistema, foi realizada a carga de 135.800 apólices. A Figura \ref{fig:fig9} apresenta a quantidade de transações efetuadas na fase de preparação dos testes.

\begin{figure}[h]
    \centering
    \includegraphics[width=1\linewidth]{images/tps.png}
    \caption{Quantidade de Transações Realizadas}
    \label{fig:fig9}
\end{figure}

Foi realizada uma consulta por identificador (ID) de apólice, apresentando um tempo de resposta satisfatório de 91,87 ms. A Figura \ref{fig:fig10} ilustra a consulta executada por meio de uma API REST, evidenciando um resultado considerado satisfatório.

\begin{figure}[h]
    \centering
    \includegraphics[width=\linewidth]{images/ConsultaPolicybyID.png}
    \caption{Resultado de Consulta de Apólice pelo ID}
    \label{fig:fig10}
\end{figure}

Posteriormente, foi efetuada uma consulta utilizando como parâmetro de filtragem o documento de identificação do titular da apólice, cujo tempo de resposta excedeu 30 segundos, resultando em falha por timeout. Esse comportamento indica que o processamento da requisição demandou um intervalo superior ao limite esperado pela aplicação cliente, deixando de atender aos critérios de desempenho previamente estabelecidos.

A Figura \ref{fig:fig11}, a seguir, apresenta o resultado da consulta realizada, com o erro gerado após superar o limite de tempo configurado na aplicação cliente.

\begin{figure}[h]
    \centering
    \includegraphics[width=1\linewidth]{images/SearchByHolderId.png}
    \caption{Resultado para busca de apólice com o Id do titular}
    \label{fig:fig11}
\end{figure}

No Cenário 3, foi realizada uma requisição por meio de API REST para a obtenção da lista de identificadores (IDs) de apólices existentes. O tempo de resposta excedeu 30 segundos, ocasionando a ocorrência de um erro de timeout na aplicação e, consequentemente, o não atendimento aos critérios de desempenho estabelecidos. A Figura \ref{fig:fig12} apresenta a requisição efetuada, bem como a mensagem de erro retornada na resposta à consulta.

\begin{figure}[h]
    \centering
    \includegraphics[width=1\linewidth]{images/SearchAll.png}
    \caption{Resultado de consulta para todas as apólices}
    \label{fig:fig12}
\end{figure}

O sistema não apresentou desempenho satisfatório na execução da consulta parametrizada pelo identificador (Id) da apólice. De modo análogo, nas consultas em que se empregou, como critério de filtragem, o documento do titular da apólice ou a requisição para que a aplicação gerasse a lista contendo o Id de todas as apólices emitidas, o sistema evidenciou não conformidade em relação aos critérios previamente estabelecidos. Diante disso, torna-se necessária a revisão do modelo, de forma a possibilitar a realização das consultas sem prejuízos operacionais, mesmo em cenários de grande volume de dados, assegurando a manutenção dos níveis de desempenho e eficiência do sistema.

\subsection{Discussão}
Esta subseção de discussão foi estruturada em duas dimensões principais: segurança e desempenho. Na dimensão de segurança, examina-se os mecanismos de proteção adotados, com ênfase no controle de fraude por meio da garantia de integridade da informação, bem como na capacidade de resiliência da aplicação diante de cenários de comprometimento. Na dimensão de desempenho, analisa-se a eficiência do processo de comercialização de apólices de seguro, especialmente no que se refere ao tempo de resposta da aplicação proposta.


\subsubsection{Segurança}
A integridade da apólice de seguros constitui requisito fundamental para assegurar a confiabilidade, a segurança e a imutabilidade das informações registradas em blockchain. No contexto da solução baseada na rede Hyperledger Fabric, foram desenvolvidos três cenários de teste com o objetivo de avaliar a capacidade do sistema de proteger a apólice contra acessos não autorizados e modificações indevidas. Esses testes foram concebidos para validar a implementação de políticas de autenticação e autorização, bem como para verificar a consistência dos dados armazenados na blockchain.

Os cenários apresentados no Table \ref{tab:tab1} descrevem de forma detalhada os procedimentos executados e os critérios de sucesso estabelecidos para cada um deles. Os testes foram estruturados de modo a abranger diferentes dimensões de segurança e proteção de dados na rede blockchain.

A execução dos testes possibilitou verificar se a aplicação mantém a segurança e a integridade das apólices de seguro, prevenindo acessos indevidos e alterações não autorizadas. A abordagem adotada assegurou que os contratos armazenados na blockchain permanecessem confiáveis, imutáveis e acessíveis apenas aos participantes devidamente autorizados no sistema.

\subsubsection{Desempenho}
A eficiência da aplicação na venda de apólices de seguro é um fator essencial para garantir a viabilidade da aplicação proposta, especialmente em ambientes que demandam alta escalabilidade e confiabilidade. Dessa forma, a análise de desempenho visa medir a capacidade da aplicação descentralizada baseada em blockchain de processar transações de forma eficiente e com tempos de resposta aceitáveis sob diferentes cenários de carga e volume de dados.

Para avaliar essa eficiência, foram definidos três cenários principais de teste, conforme detalhado na Table \ref{tab:tab2}. Esses cenários foram projetados para analisar o tempo de resposta por transação, a escalabilidade do sistema sob cargas crescentes e a capacidade da aplicação de lidar com grandes volumes de dados sem comprometer a performance.

O Cenário 1, teve como propósito a medição do tempo médio de resposta para o processamento de uma transação individual de venda de apólice, verificando se o sistema atende aos limites aceitáveis para um funcionamento eficiente. Para isso, foram criadas 1000 apólices de forma sequencial, registrando o tempo necessário para cada operação.

O Cenário 2, teve como foco a avaliação do impacto do aumento do número de transações simultâneas no tempo de resposta da aplicação. Foram simuladas cargas progressivas de 100, 500, 1000 e 2000 transações por segundo, monitorando a latência, o uso de CPU e memória, bem como o tempo de commit das transações na blockchain.

O Cenário 3, avaliou o impacto da manipulação de grandes volumes de dados na eficiência da aplicação, simulando um banco de dados com mais de 100 mil apólices. Foram realizadas consultas por ID da apólice, por titular e a listagem completa das apólices cadastradas.

Os testes evidenciaram pontos de atenção relacionados ao desempenho em consultas sobre grandes conjuntos de dados, que exigem ajustes para garantir uma performance consistente e responsiva em cenários de alta densidade de informações. Esses pontos, embora não comprometam o funcionamento principal da aplicação, são relevantes para sua adoção em ambientes de produção com alta demanda analítica.



\section{Limitações e Trabalhos Futuros}

\textbf{Limitações:} Apesar das contribuições deste estudo, algumas limitações devem ser consideradas. Em primeiro lugar, a prototipação da prova de conceito foi conduzida em um ambiente controlado, utilizando uma infraestrutura baseada em contêineres com Docker, o que pode divergir de cenários reais de produção, nos quais requisitos de escalabilidade e resiliência da rede tendem a impor desafios adicionais. Ademais, a aplicação proposta foi concebida para um caso específico de seguro-viagem, o que pode demandar adaptações para sua utilização em outros ramos de seguros, bem como para o atendimento a diferentes marcos regulatórios em distintas jurisdições. 

Outra limitação refere-se à complexidade inerente à tecnologia blockchain, que pode requerer investimentos significativos em capacitação técnica e em infraestrutura para viabilizar sua adoção em larga escala pelas organizações do setor. Por fim, aspectos relacionados à governança da rede e à integração com sistemas legados não foram explorados em profundidade neste trabalho, embora representem desafios críticos para a implementação prática da solução em ambientes corporativos reais.

\textbf{Trabalhos futuros:} Diante das limitações identificadas, diversas direções para investigações futuras podem ser delineadas. Inicialmente, a realização de experimentos em ambientes de produção ou em redes blockchain permissionadas de larga escala permitiria avaliar, de forma mais robusta, a escalabilidade e o desempenho da solução sob condições reais de operação. Adicionalmente, a adaptação da aplicação para contemplar distintos tipos de seguros, levando em consideração regulamentações específicas de diferentes jurisdições, poderia ampliar sua aplicabilidade e relevância no setor securitário.

Investigações subsequentes podem concentrar-se na proposição de estratégias de otimização do desempenho do contrato inteligente, com foco na redução de custos computacionais e na mitigação da latência das transações na rede. A incorporação de técnicas avançadas, como o emprego de inteligência artificial para análise de risco e modelagem preditiva, configura-se como uma possibilidade promissora para agregar valor à solução, aumentando sua eficiência operacional e sua capacidade de suporte à tomada de decisão.

Outra vertente de evolução envolve o aperfeiçoamento dos mecanismos de governança da rede blockchain, por meio da análise e adoção de diferentes mecanismos de consenso e modelos de gestão descentralizada, com o objetivo de definir, de maneira transparente e auditável, regras de participação e processos decisórios para o registro de transações entre múltiplos atores do ecossistema. A integração da aplicação com plataformas legadas das seguradoras também se apresenta como um campo de estudo relevante, possibilitando uma transição mais gradual e interoperável para o uso de blockchain, sem descontinuar ou comprometer fluxos de trabalho já consolidados.

No que tange à segurança em ambientes blockchain, esta constitui um requisito fundamental para assegurar a integridade e a autenticidade das transações, sobretudo em sistemas que manipulam informações sensíveis, como dados de apólices de seguro e informações pessoais de clientes. Uma potencial linha de aprimoramento consiste na implementação de mecanismos de controle de acesso mais sofisticados, incluindo a integração de sistemas de autenticação multifatorial (MFA) e a adoção de algoritmos criptográficos de última geração, de modo a proporcionar maior resiliência frente a ataques de força bruta e outras vulnerabilidades. Ademais, a ampliação do uso de criptografia de ponta a ponta (end-to-end) poderia elevar o nível de proteção dos dados em trânsito, aspecto crítico dada a natureza sensível das informações tratadas.

Por fim, avanços relacionados à realização de testes de intrusão (penetration tests) específicos para contratos inteligentes e à implementação de soluções de monitoramento contínuo e em tempo real, voltadas à detecção precoce de atividades anômalas e tentativas de fraude, têm o potencial de reforçar substancialmente a resiliência global do sistema.

Em síntese, pesquisas orientadas à melhoria do desempenho e ao fortalecimento dos mecanismos de segurança, em conjunto com o aperfeiçoamento contínuo dos controles de acesso, dos processos de auditoria e das capacidades de monitoramento, representam um avanço significativo para a consolidação da confiança, da integridade e da conformidade da solução blockchain em ambientes do setor de seguros.


\section{Conclusão}

Os objetivos estabelecidos no início desta dissertação foram integralmente alcançados. Em primeiro lugar, foi conduzida uma revisão sistemática da literatura que evidenciou o estado da arte referente a aplicações descentralizadas baseadas em blockchain para a comercialização de apólices de seguros. Essa revisão forneceu uma base conceitual e tecnológica robusta para o desenvolvimento da aplicação descentralizada proposta, destacando as principais tecnologias e frameworks disponíveis na área. Na sequência, foi especificada uma arquitetura destinada ao desenvolvimento de um processo seguro para a venda de apólices de seguros, utilizando a tecnologia blockchain como infraestrutura para garantir descentralização, persistência, anonimato e auditabilidade das transações.

A prototipagem da prova de conceito evidenciou a viabilidade do uso de blockchain, especificamente da plataforma Hyperledger Fabric, para a comercialização de apólices de seguro. A adoção de uma infraestrutura baseada em containers assegurou um ambiente controlado, isolado e reprodutível, enquanto a configuração detalhada da rede garantiu a autenticidade, integridade e segurança das transações realizadas. O contrato inteligente desenvolvido permitiu a automatização de operações essenciais do domínio de seguros, promovendo transparência e confiabilidade a todo o processo transacional. Adicionalmente, a integração com uma API REST possibilitou a comunicação eficiente entre os usuários e a rede blockchain, resultando em um sistema com potencial de robustez e escalabilidade.

Os testes funcionais executados validaram a correta implementação e execução das principais operações do sistema, incluindo a criação, consulta, atualização e cancelamento de apólices, bem como o cálculo de reembolsos. Os resultados demonstraram que as regras de negócio foram adequadamente aplicadas e que o desempenho do sistema esteve em conformidade com os requisitos e expectativas estabelecidos, assegurando a integridade, consistência e confiabilidade das transações. Dessa forma, os achados deste trabalho corroboram o potencial da tecnologia blockchain para promover inovações significativas no setor de seguros, contribuindo para processos mais eficientes, seguros e confiáveis nas transações envolvendo apólices.



\end{document}
